\chapter{Future Work}

The work outlined in previous chapters should be largely completed by the end of 2010. It may be that further investigation reveals additional avenues to explore; however, it will be necessary to find new projects as well. Development of new areas of study will depend upon what is presented in the literature in the intervening time. Current ideas are discussed below.

\section{Other Theories Of Gravity}

Analysis similar to that discussed in chapter 2 for metric $f(R)$ gravity may be performed for other theories of modified gravity. This is a rapidly developing area incorporating ideas from quantum gravity and cosmology. Other theories to be investigated could include:
\begin{itemize}
\item Metric-affine gravity\cite{Sotiriou2007, Sotiriou2007b}, as discussed in \secref{Action}. Since this is not a metric theory of gravity it may be possible to find observational tests that strongly constrain, or rule out this theory\cite{Will2006}.
\item{} Generalised higher-order gravities which replace $R$ in the Einstein-Hilbert action with $f(R, R_{\mu\nu}R^{\mu\nu}, R_{\mu\nu\rho\sigma}R^{\mu\nu\rho\sigma})$\cite{Farhoudi2006, Madsen1989}. We see that $f(R)$ is just a simplification of this case. Again we should recover the results of quadratic gravity in linearized theory\cite{Pechlaner1966, Stelle1978, Schmidt1986, Teyssandier1990, Capozziello2009b}.
\item{} Ho\v{r}ava-Lifshitz gravity\cite{Horava2009, Blas2010a, Sotiriou2009c} which sacrifices spacetime covariance in favour of being renormalizable. A preferred foliation of space and time along the lines of the Arnowitt-Deser-Misner (ADM) formalism is adopted\cite{Arnowitt1962a}, with Lorentz invariance being emergent at large distances. This removes many of the problems associated with time traditionally associated with trying to quantize GR.
\item{} Chern-Simons modified gravity\cite{Alexander2009a} which includes gravitational parity violation. Motivated by gauge theories, Chern-Simons gravity includes a term in the action proportional to the Pontryagin density ${^\ast R} R = \nicerecip{2}\,\epsilon^{\nu\rho\sigma\tau}{R^\lambda}_{\mu\sigma\tau}{R^\mu}_{\lambda\nu\rho}$, where $\epsilon^{\nu\rho\sigma\tau}$ is the Levi-Civita alternating tensor, coupled to a (pseudo-)scalar filed $\vartheta$. Consequences of this include birefringent GWs, altered precession rates, and the modification of vacuum solutions that are axisymmetric but not spherically symmetric such as Kerr.
\end{itemize}
Since there are so many ways to formulate an alternate theory of gravity, there are many opportunities for study in this area. It would be desirable to find tests that can distinguish these theories from each other and GR; strong-field tests seem the most promising.

\section{Observing Black Hole Shadows}

Black holes are intriguing objects. In the next few years it is hoped that VLBI will advance to the stage that it will be possible to resolve features of the size of the order of the event horizon for sufficiently massive galactic BHs\cite{Doeleman2008}. Due to its mass and proximity, Sgr A* is the prime candidate\cite{Broderick2009}. This capability would allow us to directly image accretion flows down to the event horizon, and would be the first direct evidence that these compact objects are actually BHs as currently understood, not some other exotic object.

One of the main targets of these strong-field VLBI observations is the measurement of the BH's shadow. This is the dark region surrounding the BH from which no light can reach the observer; it is bounded by the innermost photon orbit\cite{Chandrasekhar1998}. The exact shape of the shadow is intimately linked to the metric and is a sensitive probe of the spacetime. By measuring the shape of the shadow it may be possible to measure the spin and inclination of the BH\cite{Hioki2009a}, assuming it is Kerr, check whether it is an over-extreme Kerr BH\cite{Bambi2009}, or even probe deviations from Kerr\cite{Johannsen2010a, Johannsen2010b}. It would be interesting to investigate the shape of the shadow in other spacetimes, for example Manko-Novikov\cite{Manko1992, Gair2008a} which form a family of exact asymptotically flat spacetimes with arbitrary multipole moments. The shape of the shadow of a Kerr BH is shown in \figref{Shadow}.
\begin{figure}[htb]
  \begin{center}
   \subfigure[$a = -0.2 M_\bullet$, $\theta\sub{obs} = \pi/2$]{\resizebox{0.3\textwidth}{!}{\import{./Images/}{Shadow_2-psfrag.tex}}} \quad
   \subfigure[$a = -0.2 M_\bullet$, $\theta\sub{obs} = \pi/6$]{\resizebox{0.3\textwidth}{!}{\import{./Images/}{Shadow_2_6-psfrag.tex}}} \quad
   \subfigure[$a = 0.4 M_\bullet$, $\theta\sub{obs} = \pi/2$]{\resizebox{0.3\textwidth}{!}{\import{./Images/}{Shadow_4-psfrag.tex}}} \\
   \subfigure[$a = 0.4 M_\bullet$, $\theta\sub{obs} = \pi/6$]{\resizebox{0.3\textwidth}{!}{\import{./Images/}{Shadow_4_6-psfrag.tex}}} \quad
   \subfigure[$a = 0.9 M_\bullet$, $\theta\sub{obs} = \pi/2$]{\resizebox{0.3\textwidth}{!}{\import{./Images/}{Shadow_9-psfrag.tex}}} \quad
   \subfigure[$a = 0.9 M_\bullet$, $\theta\sub{obs} = \pi/6$]{\resizebox{0.3\textwidth}{!}{\import{./Images/}{Shadow_9_6-psfrag.tex}}} \\
   \subfigure[$a = 0.998 M_\bullet$, $\theta\sub{obs} = \pi/2$]{\resizebox{0.3\textwidth}{!}{\import{./Images/}{Shadow_998-psfrag.tex}}} \quad  
   \subfigure[$a = 0.998 M_\bullet$, $\theta\sub{obs} = \pi/6$]{\resizebox{0.3\textwidth}{!}{\import{./Images/}{Shadow_998_6-psfrag.tex}}}
    \caption{Apparent shape of the shadow of a Kerr BH viewed at infinity. $\alpha$ and $\beta$ are the position coordinates projected onto the celestial sphere, and $\theta\sub{obs}$ is the polar coordinate of the observer\cite{Chandrasekhar1998}. If $\theta\sub{obs} = 0, \pi$ we would be looking along the spin axis and would see a circular shadow.}
    \label{fig:Shadow}
  \end{center}
\end{figure}
The shadow remains near circular for spin values $a \lesssim 0.9 M_\bullet$ regardless of inclination (axisymmetry requires that the shadow is circular when looking along the rotation axis) even though the Kerr spacetime is highly non-spherically symmetric\cite{Johannsen2010b}. Observing deviations from Kerr would disprove the no-hair theorem (possibly admitting naked singularities), provide evidence for a non-GR theory of gravity, or both. In order to do so it will be necessary to find a convenient parameterization to describe the shape of the shadow.
