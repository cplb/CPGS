\chapter{Introduction}\setcounter{page}{1}

\section{Gravitation}

Gravity is one of the fundamental forces of nature. It is familiar as the force that keeps the Earth in orbit about the Sun, causes apples to fall from trees and makes falling of a log so easy. Yet there is much we do not know about gravity. We do not have a complete quantum theory, or even a definite framework to find one. Modern physics describes gravity using the classical theory of general relativity (GR). Since its inception by Einstein in 1915 GR has successfully passed every observational test. However, these test have primarily focused on weak gravitational fields. Strong gravitational fields provide more interesting tests: because gravity is stronger any correction to GR should be more noticeable. This effect is amplified because gravity is non-linear. Strong fields are found regions of high spacetime curvature, such as in the areas surrounding massive compact objects, like black holes or neutron stars.

One particularly promising method of exploring strong-field regions would be to observe gravitational waves (GWs). This would allow us to probe gravitational interactions in regimes that are currently inaccessible using more traditional, electromagnetic observations. For example, binary encounters between massive compact objects create gravitational fields that are both intensely strong and highly dynamical, a domain where GR has yet to be tested. As yet no GWs have been directly detected, although their existence has been inferred from the loss of energy and angular momentum from binary pulsars. There are a number of experiments designed to measure gravitational radiation: the ground-based detectors of the Laser Interferometer Gravitational-Wave Observatory (LIGO) and Virgo collaboration may be the first to see GWs, but of particular interest for many astrophysical applications is the planned NASA/ESA Laser Interferometer Space Antenna (LISA). Observing GWs would allow us to learn about the systems that generate them. As an example, it may be possible to infer information about the massive black hole believed to be at the centre of our own galaxy.

While GWs are an exciting source of information, it will be beneficial to compare with results from other techniques, to maximise the data available for inferences, and to check models. For example, very long baseline interferometry (VLBI) may be used to image the vicinity of a black hole's horizon, or X-ray observations could be used to investigate black hole accretion discs. This work investigates what we might be able to learn about gravity and massive compact objects through a variety of strong-field and weak-field tests, with an emphasis upon GWs.

\section{Structure}

In chapter 2 we examine an alternative theory of gravity: metric $f(R)$. We focus on the modifications to gravitational radiation and possible observational tests that may be used to constrain the theory. Many of the results are already known in the literature, but are worked out here {\it ab initio}. We include them as a compendium of useful results, within a consistent system of notation, and to highlight some important points. Seemingly new results are found in sections \ref{sec:EM_tensor} and \ref{sec:Epicycle}.

In chapter 3 we investigate what might be inferred by observing gravitational radiation from an object on a highly eccentric orbit about the galactic centre. Waveform construction, signal analysis and parameter estimation are discussed, and some preliminary results are presented.

Finally in chapter 4 we outline further areas of interest that may be studied in the future.

\section{Conventions}

Throughout this work we will use the time-like sign convention of Landau and Lifshitz\cite{Landau1975}:
\begin{enumerate}
\item The metric has signature $(+,-,-,-)$.
\item The Reinann tensor is defined as ${R^\mu}_{\nu\sigma\rho} = \partial_\sigma {\Gamma^\mu}_{\nu\rho} - \partial_\rho {\Gamma^\mu}_{\nu\sigma} + {\Gamma^\mu}_{\lambda\sigma}{\Gamma^\lambda}_{\rho\nu} - {\Gamma^\mu}_{\lambda\rho}{\Gamma^\lambda}_{\sigma\nu}$.
\item The Ricci tensor is defined as the contraction $R_{\mu\nu} = {R^\lambda}_{\mu\lambda\nu}$.
\end{enumerate}
Greek indices are used to represent spacetime indices $\mu = \{0,1,2,3\}$ and lowercase Latin indices from the middle of the alphabet are used for spatial indices $i = \{1,2,3\}$. Uppercase Latin indices from the beginning of the alphabet will be used for the output of two LISA detector arms $A = \{\mathrm{I}, \mathrm{II}\}$, and lowercase Latin indices from the beginning of the alphabet are used for parameter space. Summation over repeated indices is assumed unless explicitly noted otherwise. Geometric units with $G = c = 1$ will be used where noted, but in general factors of $G$ and $c$ will be retained.
