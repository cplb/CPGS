\chapter{Introduction}\setcounter{page}{1}

\section{Gravitation}

Detecting gravitational waves (GWs) would allow us to probe gravitational interactions in regimes that are currently inaccessible using more traditional, electromagnetic observations. For example, encounters between compact objects, like black holes or neutron stars, create gravitational fields that are both intensely strong and highly dynamical, a domain where GR has yet to be tested.

\section{Structure}

In chapter 2 we examine an alternative theory of gravity: metric $f(R)$. We focus on the modifications to gravitational radiation and possible observational tests that may be used to constrain the theory. Many of the results are already known in the literature, but are worked out here {\it ab initio}. We include them as a compendium of useful results, within a consistent system of notation, and to highlight some important points. Seemingly new results are found in sections \ref{sec:EM_tensor} and \ref{sec:Epicycle}.

In chapter 3 we investigate what might be inferred by observing gravitational radiation from an object on a highly eccentric orbit about the galactic centre. Waveform construction, signal analysis and parameter estimation are discussed, and some preliminary results are presented.

Finally in chapter 4 we outline further areas of interest that may be studied in the future.

\section{Conventions}

Throughout this work we will use the time-like sign convention of Landau and Lifshitz\cite{Landau1975}:
\begin{enumerate}
\item The metric has signature $(+,-,-,-)$.
\item The Reinann tensor is defined as ${R^\mu}_{\nu\sigma\rho} = \partial_\sigma {\Gamma^\mu}_{\nu\rho} - \partial_\rho {\Gamma^\mu}_{\nu\sigma} + {\Gamma^\mu}_{\lambda\sigma}{\Gamma^\lambda}_{\rho\nu} - {\Gamma^\mu}_{\lambda\rho}{\Gamma^\lambda}_{\sigma\nu}$.
\item The Ricci tensor is defined as the contraction $R_{\mu\nu} = {R^\lambda}_{\mu\lambda\nu}$.
\end{enumerate}
Greek indices are used to represent spacetime indices $\mu = \{0,1,2,3\}$ and lowercase Latin indices from the middle of the alphabet are used for spatial indices $i = \{1,2,3\}$. Uppercase Latin indices from the beginning of the alphabet will be used for the output of two LISA detector arms $A = \{\mathrm{I}, \mathrm{II}\}$, and lowercase Latin indices from the beginning of the alphabet are used for parameter space. Summation over repeated indices is assumed unless explicitly noted otherwise. Geometric units with $G = c = 1$ will be used where noted, but in general factors of $G$ and $c$ will be retained.
