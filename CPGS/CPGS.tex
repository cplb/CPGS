\documentclass[a4paper, 11pt, titlepage, twoside]{report}
\usepackage[a4paper]{geometry}
\usepackage{fancyhdr}
\usepackage{charter}
\usepackage{pstricks,pst-node,pst-text,pst-3d}
\usepackage{psfrag}
\usepackage{import}
\usepackage{graphicx}
\usepackage{subfigure}
\usepackage{amssymb,amstext} %% ... with default font
\usepackage{amsmath}
\usepackage{listings}
%\usepackage{tikz}
%\usetikzlibrary{shapes,arrows}
\usepackage{lmodern}
\usepackage{microtype}
\usepackage{nicefrac}
\usepackage{siunitx}
\renewcommand{\sfdefault}{lmss}
\renewcommand{\ttdefault}{lmtt}
\usepackage[T1]{fontenc}
\usepackage{braket}
\usepackage{afterpage}
\usepackage[latin9]{inputenc}
%\usepackage{txfonts}
\usepackage{natbib}
\renewcommand{\bibfont}{\footnotesize}

  \hyphenpenalty=4000
  \tolerance=1000
  
\headheight 13.6pt
\textheight = 45\baselineskip

\def\today{\number\day\space\ifcase\month\or January\or February\or March\or April\or May\or June\or July\or August\or September\or October\or November\or December\fi\space\number\year} 

\newcommand{\eqnref}[1]{equation~(\ref{eq:#1})}
\newcommand{\Eqnref}[1]{Equation~(\ref{eq:#1})}
\newcommand{\secref}[1]{section~\ref{sec:#1}}
\newcommand{\Secref}[1]{Section~\ref{sec:#1}}
\newcommand{\apref}[1]{appendix~\ref{sec:#1}}
\newcommand{\Apref}[1]{Appendix~\ref{sec:#1}}
\newcommand{\tabref}[1]{table~\ref{tab:#1}}
\newcommand{\Tabref}[1]{Table~\ref{tab:#1}}
\newcommand{\figref}[1]{figure~\ref{fig:#1}}
\newcommand{\Figref}[1]{Figure~\ref{fig:#1}}

\newcommand{\units}[1]{\ensuremath{~\mathrm{#1}}}
\newcommand{\sub}[1]{\ensuremath{_\mathrm{#1}}}

\newcommand{\recip}[1]{\ensuremath{\frac{1}{#1}}}

\newcommand{\order}[1]{\ensuremath{\mathcal{O}({#1})}}

% Define differential operators
\newcommand{\dd}{\ensuremath{\mathrm{d}}}
\newcommand{\diff}[2]{\ensuremath{\frac{\dd {#1}}{\dd {#2}}}}
\newcommand{\linediff}[2]{\ensuremath{\dd {#1}/\dd {#2}}}
\newcommand{\diffop}[1]{\ensuremath{\frac{\dd}{\dd {#1}}}}
\newcommand{\difftwo}[2]{\ensuremath{\frac{\dd^2 {#1}}{\dd {#2}^2}}}
\newcommand{\linedifftwo}[2]{\ensuremath{\dd^2 {#1}/\dd {#2}^2}}
\newcommand{\difftwoop}[1]{\ensuremath{\frac{\dd^2}{\dd {#1}^2}}}
\newcommand{\diffn}[3]{\ensuremath{\frac{\dd^{#3} {#1}}{\dd {#2}^{#3}}}}
\newcommand{\linediffn}[3]{\ensuremath{\dd^{#3} {#1}/\dd {#2}^{#3}}}
\newcommand{\diffnop}[2]{\ensuremath{\frac{\dd^{#2}}{\dd {#1}^{#2}}}}
\newcommand{\partialdiff}[2]{\ensuremath{\frac{\partial {#1}}{\partial {#2}}}}
\newcommand{\linepartialdiff}[2]{\ensuremath{\partial {#1}/\partial {#2}}}
\newcommand{\partialdiffop}[1]{\ensuremath{\frac{\partial}{\partial {#1}}}}
\newcommand{\grad}{\ensuremath{\boldsymbol{\nabla}}}
\newcommand{\intd}[4]{\ensuremath{\int_{#1}^{#2}{#3}\,\dd{#4}}}

\begin{document}

\bibpunct{[}{]}{,\,}{s}{}{} 

\pagestyle{fancy}
\renewcommand{\chaptermark}[1]{
\markboth{\chaptername
\ \thechapter.\ #1}{}}
\renewcommand{\sectionmark}[1]{\markright{\thesection\ #1}}
\fancyhf{}
\fancyhead[LE]{\leftmark}
\fancyhead[RE]{Christopher Berry}
\fancyhead[LO]{Exploring Strong Field Gravity}
\fancyhead[RO]{\rightmark}
\fancyfoot{}
\renewcommand{\footrulewidth}{0.1ex}
\fancyfoot[LO]{\today}
\fancyfoot[RE]{Institute of Astronomy, Cambridge}
\fancyfoot[RO, LE]{\thepage}
\fancypagestyle{plain}{
		\fancyhf{}
		\renewcommand{\headrulewidth}{0pt}
		\fancyfoot{}
		\renewcommand{\footrulewidth}{0.1ex}
		\fancyfoot[LO]{\today}
		\fancyfoot[RE]{Institute of Astronomy, Cambridge}
		\fancyfoot[RO, LE]{\thepage}}
\renewcommand\floatpagefraction{.6}
\renewcommand\topfraction{.8}
\renewcommand\bottomfraction{.8}
\renewcommand\textfraction{.2}

\title{Exploring Strong Field Gravity}
\author{Christopher Berry\\ \vspace{3mm} Institute of Astronomy, University of Cambridge\\ \vspace{3mm} Supervisor: Jonathan Gair}
\date{\today}
\maketitle

\begin{abstract}
Abstract
\end{abstract}

\chapter{Introduction}\setcounter{page}{1}
\section{Introduction}

'Lo

\newpage

Hi

\chapter{Gravitational Radiation In $f(R)$ Theory}

\section{The Action \& Field Equations}

In $f(R)$ we modify the standard Einstein-Hilbert action
\begin{equation}
S[g] = \frac{c^4}{16\pi G}\intd{}{}{R\sqrt{-g}}{^4x},
\end{equation}
to include an arbitrary function of the Ricci scalar $R$ such that
\begin{equation}
S[g] = \frac{c^4}{16\pi G}\intd{}{}{f(R)\sqrt{-g}}{^4x}.
\end{equation}
We will assume that $f(R)$ is analytic about $R = 0$ so that it may be expressed as a power series
\begin{equation}
f(R) = a_0 + a_1 R + \frac{a_2}{2!}R^2 + \frac{a_3}{3!}R^3 + \ldots
\end{equation}
Note that since the dimensions of $f(R)$ must be the same as of $R$, $[a_n] = [R]^{(1-n)}$. To link to GR we will set $a_1 = 1$, any rescaling may be absorbed into the definition of $G$.

We may calculate the field equations in $f(R)$ theory by varying the action with respect to the metric $g^{\mu\nu}$,
\begin{align}
\delta S = {} & \frac{c^4}{16\pi G}\intd{}{}{\frac{\delta}{\delta g^{\mu\nu}}(f(R)\sqrt{-g})\delta g^{\mu\nu}}{^4x}\nonumber \\
 = {} & \frac{c^4}{16\pi G}\intd{}{}{\left\{f'(R)\frac{\delta R}{\delta g^{\mu\nu}}\sqrt{-g} - f(R)\frac{1}{2\sqrt{-g}}\frac{\delta g}{\delta g^{\mu\nu}}\right\}\delta g^{\mu\nu}}{^4x},
\label{eq:delta_S}
\end{align}
where we have used the Liebniz and chain rules, and a prime indicates differentiation with respect to $R$. We will consider each term individually.

The variation of the determinant gives, by consideration of the cofactor of $g^{\mu\nu}$,
\begin{align}
\delta g = {} & gg^{\mu\nu}\delta g_{\mu\nu}\nonumber \\
 = {} & -gg_{\mu\nu}\delta g^{\mu\nu}.
 \label{eq:delta g}
\end{align}
The second line follows from
\begin{equation}
\delta \delta^\mu_\nu = 0
\end{equation}
where the Kronecker represents $\delta^\mu_\nu = g^{\mu\lambda}g_{\lambda\nu}$.

The variation of the Ricci scalar is more involved. We start with variation in the Riemann tensor
\begin{align}
\delta {R^\rho}_{\sigma\mu\nu} = {} & \partial_\mu\delta{\Gamma^\rho}_{\nu\sigma} - \partial_\nu\delta{\Gamma^\rho}_{\mu\sigma} + \delta{\Gamma^\rho}_{\lambda\mu}{\Gamma^\lambda}_{\nu\sigma} + {\Gamma^\rho}_{\lambda\mu}\delta{\Gamma^\lambda}_{\nu\sigma}\nonumber \\
 & - \delta{\Gamma^\rho}_{\lambda\nu}{\Gamma^\lambda}_{\mu\sigma} - {\Gamma^\rho}_{\lambda\nu}\delta{\Gamma^\lambda}_{\mu\sigma}\nonumber \\
 = {} & \nabla_\mu\delta{\Gamma^\rho}_{\nu\sigma} - \nabla_\nu\delta{\Gamma^\rho}_{\mu\sigma}.
\end{align}
Thus, contracting over indices, the variation in the Ricci tensor is
\begin{align}
\delta R_{\mu\nu} = {} & \delta {R^\rho}_{\mu\rho\nu} \nonumber \\
 = {} & \nabla_\rho\delta{\Gamma^\rho}_{\nu\mu} - \nabla_\rho\delta{\Gamma^\nu}_{\rho\mu}.
\end{align}
Using this, the variation in the Ricci scalar is
\begin{align}
\delta R = {} & R_{\mu\nu}\delta g^{\mu\nu} + g^{\mu\nu}\delta R_{\mu\nu} \nonumber \\
 = {} & R_{\mu\nu}\delta g^{\mu\nu} + g^{\mu\nu}\left(\nabla_\rho\delta{\Gamma^\rho}_{\nu\mu} - \nabla_\rho\delta{\Gamma^\nu}_{\rho\mu}\right).
\end{align}
We just need the variation in the connection coefficients
\begin{align}
\delta {\Gamma^\rho}_{\mu\nu} = {} & \frac{1}{2}\left[\delta g^{\rho\lambda}(\partial_\mu g_{\lambda\nu} + \partial_\nu g_{\lambda\mu} - \partial_\lambda g_{\mu\nu})\right.\nonumber \\
 & + \left. g^{\rho\lambda}(\partial_\mu \delta g_{\lambda\nu} + \partial_\nu \delta g_{\lambda\mu} - \partial_\lambda \delta g_{\mu\nu})\right]\nonumber \\
 = {} & \frac{1}{2}\left[\delta g^{\rho\lambda}(\partial_\mu g_{\lambda\nu} + \partial_\nu g_{\lambda\mu} - \partial_\lambda g_{\mu\nu})\right.\nonumber \\
 & + \left. g^{\rho\lambda}(\nabla_\mu \delta g_{\lambda\nu} + \nabla_\nu \delta g_{\lambda\mu} - \nabla_\lambda \delta g_{\mu\nu})\right. \nonumber \\
 & + \left. g^{\rho\lambda}g^{\sigma\tau}(\partial_\mu g_{\tau\nu} + \partial_\nu g_{\tau\mu} - \partial_\tau \delta g_{\mu\nu})g_{\lambda\sigma}\right].
\end{align}
Identifying $g^{\rho\lambda}g^{\sigma\tau}\delta g_{\lambda\sigma} = -\delta g^{\rho\tau}$, this simplifies to
\begin{equation}
\delta {\Gamma^\rho}_{\mu\nu} = \frac{1}{2}g^{\rho\lambda}\left(\nabla_\mu \delta g_{\lambda\nu} + \nabla_\nu \delta g_{\lambda\mu} - \nabla_\lambda \delta g_{\mu\nu}\right)
\end{equation}
Note that this, as the difference between two connections, is a tensor. Armed with this, we find that
\begin{align}
\delta R = {} & R_{\mu\nu}\delta g^{\mu\nu} + \nabla^\mu\nabla^\nu\delta g_{\mu\nu} - g^{\mu\nu}\Box\delta g_{\mu\nu} \nonumber \\
 = {} & R_{\mu\nu}\delta g^{\mu\nu} - \nabla_\mu\nabla_\nu\delta g^{\mu\nu} + g_{\mu\nu}\Box\delta g^{\mu\nu},
\label{eq:delta R}
\end{align}
where the d'Alembertian operator is $\Box = g_{\mu\nu}\nabla^\mu\nabla^\nu$.

Inserting equations (\ref{eq:delta g}) and (\ref{eq:delta R}) into (\ref{eq:delta_S}) yields
\begin{equation}
\delta S = \frac{c^4}{16\pi G}\intd{}{}{\left\{f'(R)\sqrt{-g}\left[R_{\mu\nu} - \nabla_\mu\nabla_\nu\ + g_{\mu\nu}\Box\right] - f(R)\frac{1}{2\sqrt{-g}}gg_{\mu\nu}\right\}\delta g^{\mu\nu}}{^4x}.
\end{equation}
Integrating by parts and assuming that the variation $\delta g^{\mu\nu}$ vanishes at infinity allows this to be rewritten as
\begin{equation}
\delta S = \frac{c^4}{16\pi G}\intd{}{}{\left\{f'(R)R_{\mu\nu} - \nabla_\mu\nabla_\nu f'(R) + g_{\mu\nu}\Box f'(R) - f(R)\frac{1}{2}g_{\mu\nu}\right\}\delta g^{\mu\nu}\sqrt{-g}}{^4x}.
\end{equation}
For the action to be extremised for, $\delta S = 0$, for any variation $\delta g^{\mu\nu}$, the term in braces must vanish. Thus we obtain the vacuum field equation
\begin{equation}
f'R_{\mu\nu} - \nabla_\mu\nabla_\nu f' + g_{\mu\nu}\Box f' - \frac{f}{2}g_{\mu\nu} = 0.
\label{eq:Field_eq}
\end{equation}
For standard GR, when $f(R) = R$, this reduces to the familiar
\begin{equation}
R_{\mu\nu} - \frac{R}{2}g_{\mu\nu} = 0.
\end{equation}
Taking the trace of our field equation gives
\begin{equation}
f'R + 3\Box f' - 2f = 0.
\label{eq:Trace_eq}
\end{equation}
Note if we consider a uniform flat spacetime $R = 0$, this equation gives
\begin{equation}
a_0 = 0.
\label{eq:a_0}
\end{equation}

If we introduce matter with a stress-energy tensor $T_{\mu\nu}$, the field equation becomes
\begin{equation}
f'R_{\mu\nu} - \nabla_\mu\nabla_\nu f' + g_{\mu\nu}\Box f' - \frac{f}{2}g_{\mu\nu} = \frac{8\pi G}{c^4}T_{\mu\nu}.
\end{equation}
If we act upon this with the covariant derivative we obtain
\begin{align}
\frac{8\pi G}{c^4}\nabla^\mu T_{\mu\nu} = {} & (\nabla^\mu f')R_{\mu\nu} + f'\nabla^\mu R_{\mu\nu} - \Box\nabla_\nu f' + \nabla_\nu\Box f' - \frac{f'}{2}\nabla^\mu R g_{\mu\nu} \nonumber \\
= {} & R_{\mu\nu}\nabla^\mu f' + f'\nabla^\mu\left(R_{\mu\nu} - \recip{2}R g_{\mu\nu}\right) - \left(\Box\nabla_\nu - \nabla_\nu\Box\right)f'.
\end{align}
The second term contains the covariant derivative of the Einstein tensor and so is zero. Looking at the final term
\begin{align}
\left(\Box\nabla_\nu - \nabla_\nu\Box\right)f' = {} & g^{\mu\sigma}\left[\nabla_\mu\nabla_\sigma\nabla_\nu - \nabla_\nu\nabla_\mu\nabla_\sigma\right]f' \nonumber \\
= {} & g^{\mu\sigma}\left[\partial_\mu\nabla_\sigma\partial_\nu - {\Gamma^\rho}_{\nu\mu}\nabla_\sigma\partial_\rho - {\Gamma^\rho}_{\sigma\mu}\nabla_\rho\partial_\nu\right. \nonumber \\
 & \left. - \partial_\nu\nabla_\mu\partial_\sigma + {\Gamma^\rho}_{\sigma\nu}\nabla_\mu\partial_\rho + {\Gamma^\rho}_{\mu\nu}\nabla_\rho\partial_\sigma\right]f' \nonumber \\
= {} & g^{\mu\sigma}\left[\partial_\mu\left(\partial_\sigma\partial_\nu - {\Gamma^\tau}_{\nu\sigma}\partial_\tau\right) - {\Gamma^\rho}_{\sigma\mu}\left(\partial_\rho\partial_\nu - {\Gamma^\tau}_{\nu\rho}\partial_\tau\right)\right. \nonumber \\
 & \left. - \partial_\nu\left(\partial_\mu\partial_\sigma - {\Gamma^\tau}_{\sigma\mu}\partial_\tau\right) + {\Gamma^\rho}_{\mu\nu}\left(\partial_\rho\partial_\sigma - {\Gamma^\tau}_{\sigma\rho}\partial_\tau\right)\right]f' \nonumber \\
= {} & g^{\mu\sigma}\left[\partial_\nu{\Gamma^\tau}_{\sigma\mu} -\partial_\mu{\Gamma^\tau}_{\nu\sigma} + {\Gamma^\rho}_{\sigma\mu} {\Gamma^\tau}_{\nu\rho} - {\Gamma^\rho}_{\mu\nu}{\Gamma^\tau}_{\sigma\rho}\right]\partial_\tau f' \nonumber \\
= {} & g^{\mu\sigma}{R^\tau}_{\sigma\nu\mu}\partial_\tau f' \nonumber \\
= {} & R_{\tau\nu}\nabla^\tau f',
\end{align}
which is a nice geometric identity. Using this we find that
\begin{align}
\frac{8\pi G}{c^4}\nabla^\mu T_{\mu\nu} = {} & R_{\mu\nu}\nabla^\mu f' - R_{\mu\nu}\nabla^\mu f' \nonumber \\
 = {} & 0.
\end{align}
Consequently we see that energy-momentum is a conserved quantity in the same way as in GR as may be expected from the symmetries of the action.

\section{Linear Perturbations}\label{sec:Lin}

We will now consider the case that the metric is perturbed slightly from flat Minkowski such that
\begin{equation}
g_{\mu\nu} = \eta_{\mu\nu} + h_{\mu\nu},
\end{equation}
where more formally we mean that $h_{\mu\nu} = \epsilon H_{\mu\nu}$ for small parameter $\epsilon$. We will consider terms only to $\order{\epsilon}$. Thus, the inverse metric is
\begin{equation}
g^{\mu\nu} = \eta^{\mu\nu} - h^{\mu\nu},
\end{equation}
where we have used the Minkowski metric to raise the indices, effectively defining
\begin{equation}
h^{\mu\nu} = \eta^{\mu\sigma}\eta^{\nu\rho}h_{\sigma\rho}.
\end{equation}
Similarly, the trace $h$ is given by
\begin{equation}
h = \eta^{\mu\nu}h_{\mu\nu}.
\end{equation}
This means that all quantities denoted by ``$h$'' are strictly $\order{\epsilon}$. We will have to careful later on to distinguish between quantities where the Minkowski metric has been used and those where the full metric has been used.

The linearized ($\order{\epsilon}$) connection coefficient is
\begin{equation}
{\Gamma^\rho}_{\mu\nu} = \frac{1}{2}\eta^{\rho\lambda}(\partial_\mu h_{\lambda\nu} + \partial_\nu h_{\lambda\mu} - \partial_\lambda h_{\mu\nu}).
\label{eq:Lin_Gamma}
\end{equation}
The covariant derivative of any perturbed quantity will be the same as the partial derivative. The Riemann tensor is
\begin{equation}
{R^\lambda}_{\mu\nu\rho} = \frac{1}{2}(\partial_\mu\partial_\nu h^\lambda_\rho + \partial^\lambda\partial_\rho h_{\mu\nu} - \partial_\mu\partial_\rho h^\lambda_\nu - \partial^\lambda\partial_\nu h_{\mu\rho}),
\label{eq:Lin_Riemann}
\end{equation}
where we have raised the index on the differential operator with the background Minkowski metric. Contracting gives the Ricci tensor
\begin{equation}
R_{\mu\nu} = \frac{1}{2}(\partial_\mu\partial_\rho h^\rho_\nu + \partial_\nu\partial_\rho h^\rho_\mu -\Box h_{\mu\nu} - \partial_\mu\partial_\nu h),
\label{eq:Ricci}
\end{equation}
where the d'Alembertian operator is $\Box = \eta^{\mu\nu}\partial_\mu\partial_\nu$. Contracting this with $\eta^{\mu\nu}$ we find the first order Ricci scalar
\begin{equation}
R = \partial_\mu\partial_\rho h^{\rho\mu} - \Box h.
\label{eq:Scalar}
\end{equation}

Since $R$ is $\order{\epsilon}$ we may write $f(R)$ as a Maclaurin series to first order such that
\begin{align}
f(R) = {} & a_0 + R\\
f'(R) = {} & 1 + a_2 R.
\end{align}
As we are perturbing from a flat Minkowski background where the Ricci scalar vanishes, we may use \eqnref{a_0} to set $a_0 = 0$. Inserting these into \eqnref{Field_eq} and retaining terms to first order we obtain
\begin{equation}
R_{\mu\nu} - \partial_\mu\partial_\nu(a_2 R) + \eta_{\mu\nu}\Box(a_2 R) - \frac{R}{2}\eta_{\mu\nu} = 0.
\end{equation}
We see that we need to find a relation between $R$ and its derivatives. Let us consider the linearized \eqnref{Trace_eq}
\begin{align}
R + 3 \Box(a_2 R) - 2 R  = {} & 0 \nonumber \\
3a_2 \Box R - R = {} & 0
\label{eq:Box_R}
\end{align}
This is the massive Klein-Gordon equation
\begin{equation}
\Box R + m^2R = 0,
\end{equation}
if we define mass
\begin{equation}
m^2 = -\recip{3a_2}.
\end{equation}
For a physically meaningful solution we require $m^2 > 0$, thus we constrain $f(R)$ such that $a_2 < 0$. Now substituting for $\Box R$ in the field equation gives
\begin{align}
R_{\mu\nu} - a_2 \partial_\mu\partial_\nu R + \frac{R}{3}\eta_{\mu\nu} - \frac{R}{2}\eta_{\mu\nu} = {} & 0 \nonumber \\
R_{\mu\nu} - a_2\partial_\mu\partial_\nu R - \frac{R}{6}\eta_{\mu\nu} = {} & 0.
\label{eq:Field}
\end{align}

The next step is to substitute in $h_{\mu\nu}$ to try to find wave solutions. We hope to find a quantity $\overline{h}_{\mu\nu}$ that will satisfy a wave equation, where it is related to $h_{\mu\nu}$ by
\begin{equation}
\overline{h}_{\mu\nu} = h_{\mu\nu} + A_{\mu\nu}.
\end{equation}
In GR we use the trace reversed form where $A_{\mu\nu} = -(h/2)\eta_{\mu\nu}$. This will not suffice here as we have additional terms, but let us look for a similar solution
\begin{equation}
\overline{h}_{\mu\nu} = h_{\mu\nu} - \frac{h}{2}\eta_{\mu\nu} + B_{\mu\nu}.
\end{equation}
The only rank two tensors in our theory are: $h_{\mu\nu}$, $\eta_{\mu\nu}$, $R_{\mu\nu}$, and $\partial_\mu\partial_\nu$; $h_{\mu\nu}$ has been used already, and we wish to eliminate $R_{\mu\nu}$, so we will try the simpler option based around $\eta_{\mu\nu}$. We want $B_{\mu\nu}$ to be $\order{\epsilon}$, there are three scalar quantities that satisfy this: $h$, $R$ and $\Box R$; $h$ is used already and $\Box R$ is related to $R$ by \eqnref{Box_R}. Therefore, we may construct an ansatz
\begin{equation}
\overline{h}_{\mu\nu} = h_{\mu\nu} + \left(b a_2 R - \frac{h}{2}\right)\eta_{\mu\nu}
\label{eq:Ansatz}
\end{equation}
where $a_2$ has been included to ensure dimensional consistency, and $b$ is a dimensionless number. Contracting with the background metric yields
\begin{equation}
\overline{h} = 4b a_2 R - h,
\label{eq:h_trace}
\end{equation}
so we may eliminate $h$ in our definition of $\overline{h}_{\mu\nu}$ to give
\begin{equation}
h_{\mu\nu} = \overline{h}_{\mu\nu} + \left(b a_2 R -\frac{\overline{h}}{2}\right)\eta_{\mu\nu}.
\end{equation}
We will assume a Lorenz, or de Donder, gauge choice such that
\begin{equation}
\nabla^\mu \overline{h}_{\mu\nu} = 0,
\label{eq:Lorenz}
\end{equation}
to first order this gives
\begin{equation}
\partial^\mu \overline{h}_{\mu\nu} = 0.
\end{equation}
Subject to this, the Ricci tensor is, from \eqnref{Ricci},
\begin{align}
R_{\mu\nu} = {} & \frac{1}{2}\left\{\partial_\nu\partial_\mu \left(b a_2 R -\frac{\overline{h}}{2}\right) + \partial_\mu\partial_\nu \left(b a_2 R -\frac{\overline{h}}{2}\right) \right. \nonumber \\
 & \left. - \Box \left[\overline{h}_{\mu\nu} + \left(b a_2 R -\frac{\overline{h}}{2}\right)\eta_{\mu\nu}\right] - \partial_\mu\partial_\nu (4b a_2 R - \overline{h})\right\} \nonumber \\
 = {} & -\frac{1}{2}\left\{2b a_2 \partial_\mu\partial_\nu R + \Box\left(\overline{h}_{\mu\nu} -\frac{\overline{h}}{2}\eta_{\mu\nu}\right) + \frac{b}{3}R\eta_{\mu\nu}\right\}.
\end{align}
Using this in our field equation \eqref{eq:Field} gives
\begin{equation}
\frac{1}{2}\Box\left(\overline{h}_{\mu\nu} - \frac{\overline{h}}{2}\right) + (b + 1)\left(a_2\partial_\mu\partial_\nu R + \frac{R}{6}\eta_{\mu\nu}\right) = 0.
\end{equation}
If we set $b = -1$, then the second term vanishes. Let us now consider the Ricci scalar in the case $b = -1$, then from \eqnref{Scalar}
\begin{align}
R = {} & \Box \left(a_2 R -\frac{\overline{h}}{2}\right) - \Box (-4 a_2 R - \overline{h}) \nonumber \\
 = {} & 3a_2 \Box R + \frac{1}{2}\Box \overline{h}.
\label{eq:Ricci_Box_h}
\end{align}
For consistency with \eqnref{Box_R}, we see that
\begin{equation}
\Box \overline{h} = 0.
\label{eq:Box_h}
\end{equation}
This means finally that
\begin{equation}
\Box \overline{h}_{\mu\nu} = 0,
\label{eq:Box_hmunu}
\end{equation}
we have our wave equation and it is consistent.

Should $a_2$ be sufficiently small that it may be regarded an $\order{\epsilon}$ quantity then we recover GR to leading within our analysis.

\section{Gravitational Radiation}

Having established two wave equations, \eqref{eq:Box_R} and \eqref{eq:Box_hmunu}, we may now investigate their solutions. We will use natural units $c = \hbar = 1$. Using a standard Fourier decomposition
\begin{align}
\overline{h}_{\mu\nu} = {} & \Re\left\{\hat{\overline{h}}_{\mu\nu}(k_\rho) \exp\left(ik_\rho x^\rho\right)\right\},\\
R = {} & \Re\left\{\hat{R}(q_\rho) \exp\left(iq_\rho x^\rho\right)\right\},
\end{align}
where $k_\mu$ and $q_\mu$ are the 4-wavevectors, or 4-momenta, of the waves. From \eqnref{Box_hmunu} we know that $k_\mu$ is a null vector, so for a wave traveling along the $z$-axis
\begin{equation}
k^\mu = \omega(1, 0, 0, 1),
\end{equation}
where $\omega$ is the frequency. Similarly, from \eqnref{Box_R}
\begin{equation}
q^\mu = (\Omega, 0, 0, \sqrt{\Omega^2 - m^2}),
\label{eq:Ricci_q}
\end{equation}
for frequency $\Omega$. This means that these waves do not travel at $c$, but have a group velocity
\begin{equation}
v = \frac{\sqrt{\Omega^2 - m^2}}{\Omega}.
\end{equation}
Provided that $m^2 > 0$, $v < c = 1$, but we do not get propagating modes for $\Omega^2 < m^2$.

From the condition \eqref{eq:Lorenz} we find that $k^\mu$ is orthogonal to $\hat{\overline{h}}_{\mu\nu}$,
\begin{equation}
k^\mu\hat{\overline{h}}_{\mu\nu} = 0,
\end{equation}
thus in this case
\begin{equation}
\hat{\overline{h}}_{0\nu} + \hat{\overline{h}}_{3\nu} = 0.
\label{eq:Transverse}
\end{equation}

Let us now consider the implications of \eqnref{Box_h} using equations \eqref{eq:h_trace} and \eqref{eq:Box_R},
\begin{align}
\Box\left(4a_2R + h\right) = {} & 0 \nonumber \\
\Box h = {} & -\frac{4}{3}R.
\end{align}
For non-zero $R$ (as required for the Ricci mode) there is no way we can make such that the trace $h$ will vanish. This is distinct from the case in GR. It is possible, however, to make a gauge choice such that the trace $\overline{h}$ will vanish. Consider a gauge transformation generated by $\xi_\mu$ which satisfies $\Box \xi_\mu = 0$, and so has a Fourier decomposition
\begin{equation}
\xi_\mu = \hat{\xi}_\mu \exp\left(ik_\rho x^\rho\right).
\end{equation}
We see that a transformation
\begin{equation}
\overline{h}_{\mu\nu} \rightarrow \overline{h}_{\mu\nu} + \partial_\mu\xi_\nu + \partial_\nu\xi_\mu - \eta_{\mu\nu}\partial^\rho\xi_\rho,
\end{equation}
would ensure both conditions \eqref{eq:Lorenz} and \eqref{eq:Box_hmunu} are satisfied. Under such a transformation
\begin{equation}
\hat{\overline{h}}_{\mu\nu} \rightarrow \hat{\overline{h}}_{\mu\nu} + i\left(k_\mu\hat{\xi}_\nu + k_\nu\hat{\xi}_\mu - \eta_{\mu\nu}k^\rho\hat{\xi}_\rho\right).
\end{equation}
We may therefore impose four further constraints (one for each $\hat{\xi}_\mu$) upon $\hat{\overline{h}}_{\mu\nu}$, and we may take these to be
\begin{equation}
\hat{\overline{h}}_{0\nu} = 0, \quad \hat{\overline{h}} = 0.
\end{equation}
This may appear to be five constraints, however we have already imposed \eqref{eq:Transverse}, and so setting $\hat{\overline{h}}_{00} = 0$ automatically implies $\hat{\overline{h}}_{03} = 0$.

We see that $\overline{h}_{\mu\nu}$ behaves just as its counterpart in GR so we may define
\begin{equation}
\left[\hat{\overline{h}}_{\mu\nu}\right] =
\begin{bmatrix}
0 & 0 & 0 & 0\\
0 & h_+ & h_\times & 0\\
0 & h_\times & -h_+ & 0\\
0 & 0 & 0 & 0
\end{bmatrix},
\end{equation}
where $h_+$ and  $h_\times$ are appropriate constants representing the amplitudes of the two transverse polarizations of gravitational radiation.

Using this choice of gauge we have
\begin{align}
h_{\mu\nu} = {} & \overline{h}_{\mu\nu} - a_2 R\eta_{\mu\nu},\\
h = {} & -4a_2R.
\label{eq:gauge}
\end{align}
Combining both wave modes ($\overline{h}_{\mu\nu}$ and $R$) the perturbation to the flat background associated with our wave solutions looks like
\begin{equation}
\left[{h}_{\mu\nu}(z,t)\right] =
\begin{bmatrix}
-a_2\hat{R}(t - vz) & 0 & 0 & 0\\
0 & h_+(t - z) + a_2\hat{R}(t - vz) & h_\times(t - z) & 0\\
0 & h_\times(t - z) & -h_+ + a_2\hat{R}(t - vz) & 0\\
0 & 0 & 0 & a_2\hat{R}(t - vz)
\end{bmatrix}.
\end{equation}
If we now assume that both are excited by the same source so that the frequencies will be equal $\omega = \Omega$.

It is important that our solutions reduce to those of GR in the event that $f(R) = R$. In this linearized approach this corresponds to $a_2 \rightarrow 0$, $m^2 \rightarrow \infty$. We see from \eqnref{Ricci_q} that in this limit it would take an infinite frequency to excite a propagating Ricci mode, and evanescent waves would decay away infinitely quickly. Therefore there would not be any detectable Ricci modes and we would only observe the two polarizations found in the analysis of GR. Additionally $\overline{h}_{\mu\nu}$ would simplify to its usual trace-reversed form.

\section{Energy-momentum Tensor}

We expect that the gravitational field would carry energy-momentum. Unfortunately it is difficult to define a proper energy-momentum tensor for a gravitational field: as a consequence of the equivalence principle it is possible to transform to a freely falling frame, eliminating the gravitational field and any associated energy density for a given event, although we may still define curvature in the neighbourhood of this point. We see that the full field equation, \eqnref{Field_eq}, has no energy-momentum tensor for the gravitational field on the right-hand side. However, by expanding beyond the linear terms we may find a suitable energy-momentum pseudotensor for gravitational radiation. Let us first define, in analogy to the Einstein tensor
\begin{equation}
\mathcal{G}_{\mu\nu} = f'R_{\mu\nu} - \nabla_\mu\nabla_\nu f' + g_{\mu\nu}\Box f' - \frac{f}{2}g_{\mu\nu},
\end{equation}
so that in a vacuum
\begin{equation}
\mathcal{G}_{\mu\nu} = 0.
\end{equation}
We may expand $\mathcal{G}_{\mu\nu}$ in orders of $h_{\mu\nu}$
\begin{equation}
\mathcal{G}_{\mu\nu} = {\mathcal{G}^{(\mathrm{B})}}_{\mu\nu} + {\mathcal{G}^{(1)}}_{\mu\nu} + {\mathcal{G}^{(2)}}_{\mu\nu} + \ldots
\label{eq:G_exp}
\end{equation}
We use $(\mathrm{B})$ for the background term instead of $(0)$ to avoid confusion regarding its order in $\epsilon$. Our linearised equation would then read
\begin{equation}
{\mathcal{G}^{(1)}}_{\mu\nu} = 0.
\end{equation}
So far we have assumed that our background is flat, however we can imagine that should the gravitational radiation carry energy-momentum then this would act as a source of curvature for the background. This is a second-order effect that may be encoded, to accuracy of $\order{\epsilon^2}$, as
\begin{equation}
{\mathcal{G}^{(\mathrm{B})}}_{\mu\nu} = -{\mathcal{G}^{(2)}}_{\mu\nu}.
\end{equation}
By shifting ${\mathcal{G}^{(2)}}_{\mu\nu}$ to the right-hand side we effectively create an energy-momentum tensor. As in GR we will average over several wavelengths, assuming that the background curvature is on a larger scale,
\begin{equation}
{\mathcal{G}^{(\mathrm{B})}}_{\mu\nu} = -\left\langle{\mathcal{G}^{(2)}}_{\mu\nu}\right\rangle.
\end{equation}
By doing this, we may probe the curvature in a macroscopic region about a given point in spacetime. This gives a gauge invariant measure of the gravitational field strength. The averaging can be thought of as smoothing out the rapidly varying ripples of the radiation, leaving only the coarse-grained component that acts as a source for the background curvature. The energy-momentum pseudotensor for the radiation may be identified as
\begin{equation}
t_{\mu\nu} = -\frac{c^4}{8\pi G}\left\langle{\mathcal{G}^{(\mathrm{2})}}_{\mu\nu}\right\rangle.
\end{equation}

Having made this provisional identification, we must now set about carefully evaluating the various terms in \eqnref{G_exp}. We begin as in \secref{Lin} by defining a total metric
\begin{equation}
g_{\mu\nu} = \gamma_{\mu\nu} + h_{\mu\nu},
\end{equation}
where $\gamma_{\mu\nu}$ is our background metric. This is changing slightly our definition for $h_{\mu\nu}$: instead of it being the total perturbation from flat Minkowski, it is instead the dynamical part of the metric with which we associate radiative effects. Since we know that ${\mathcal{G}^{(\mathrm{B})}}_{\mu\nu}$ is $\order{\epsilon^2}$, we may decompose our background metric as
\begin{equation}
\gamma_{\mu\nu} = \eta_{\mu\nu} + j_{\mu\nu},
\end{equation}
where $j_{\mu\nu}$ is $\order{\epsilon^2}$ to ensure that ${{R^{(\mathrm{B})}}^\lambda}_{\mu\nu\rho}$ is also $\order{\epsilon^2}$. Therefore its introduction will make no difference to the linearized theory.

We will consider terms only to $\order{\epsilon^2}$. The connection coefficient is
\begin{equation}
{\Gamma^\rho}_{\mu\nu} = {{\Gamma^{(\mathrm{B})}}^\rho}_{\mu\nu} + {{\Gamma^{(1)}}^\rho}_{\mu\nu} + {{\Gamma^{(2)}}^\rho}_{\mu\nu} + \ldots
\end{equation}
We identify ${{\Gamma^{(1)}}^\rho}_{\mu\nu}$ from \eqnref{Lin_Gamma} to the accuracy of our analysis. There is one small subtlety: whether we use the background metric $\gamma^{\mu\nu}$ or $\eta^{\mu\nu}$ to raise indices of $\partial_\mu$ and $h_{\mu\nu}$. Fortunately to the accuracy considered here it does not make a difference, however we will consider the indices to be changed with the background metric. This is more appropriate for considering the effect of curvature on gravitational radiation. We will not distinguish between $\partial_\mu$ and ${\nabla^{(\mathrm{B})}}_\mu$, the covariant derivative for the background metric: note that to the order of accuracy considered here covariant derivatives would commute and ${\nabla^{(\mathrm{B})}}_\mu$ behaves just like $\partial_\mu$. The connection coefficient has
\begin{align}
{{\Gamma^{(1)}}^\rho}_{\mu\nu} = {} & \frac{1}{2}\gamma^{\rho\lambda}\left[\partial_\mu \left(\overline{h}_{\lambda\nu} - a_2 R^{(1)}\gamma_{\lambda\nu}\right) + \partial_\nu \left(\overline{h}_{\lambda\mu} - a_2 R^{(1)}\gamma_{\lambda\mu}\right) \right. \nonumber \\ 
  & - \left. \partial_\lambda \left(\overline{h}_{\mu\nu} - a_2 R^{(1)}\gamma_{\mu\nu}\right)\right],
\end{align}
where $R^{(1)}$ is the first order Ricci scalar, and
\begin{align}
{{\Gamma^{(2)}}^\rho}_{\mu\nu} = {} & -\frac{1}{2}h^{\rho\lambda}(\partial_\mu h_{\lambda\nu} + \partial_\nu h_{\lambda\mu} - \partial_\lambda h_{\mu\nu}) \nonumber \\ 
 = {} & -\frac{1}{2}\left(\overline{h}^{\rho\lambda} - a_2 R^{(1)}\gamma^{\rho\lambda}\right)\left[\partial_\mu \left(\overline{h}_{\lambda\nu} - a_2 R^{(1)}\gamma_{\lambda\nu}\right) + \partial_\nu \left(\overline{h}_{\lambda\mu} - a_2 R^{(1)}\gamma_{\lambda\mu}\right) \right. \nonumber \\
 & - \left. \partial_\lambda \left(\overline{h}_{\mu\nu} - a_2 R^{(1)}\gamma_{\mu\nu}\right)\right].
\end{align}
The Riemann tensor is
\begin{equation}
{R^\lambda}_{\mu\nu\rho} = {{R^{(\mathrm{B})}}^\lambda}_{\mu\nu\rho} + {{R^{(1)}}^\lambda}_{\mu\nu\rho} + {{R^{(2)}}^\lambda}_{\mu\nu\rho} + \ldots
\end{equation}
We may use our expression from \eqnref{Lin_Riemann} for ${{R^{(1)}}^\lambda}_{\mu\nu\rho}$. Contracting gives the Ricci tensor
\begin{equation}
{R}_{\mu\nu} = {R^{(\mathrm{B})}}_{\mu\nu} + {R^{(1)}}_{\mu\nu} + {R^{(2)}}_{\mu\nu} + \ldots
\end{equation}
We may again use our linearized expression, \eqnref{Ricci}, for the first order term.
\begin{align}
{R^{(1)}}_{\mu\nu} = {} & \frac{1}{2}\left[\partial_\mu\partial_\rho \left(\overline{h}^\rho_\nu - a_2 R^{(1)}\gamma^\rho_\nu\right) + \partial_\nu\partial_\rho \left(\overline{h}^\rho_\mu - a_2 R^{(1)}\gamma^\rho_\mu\right) \right. \nonumber \\ 
 & - \left. \Box^{(\mathrm{B})} \left(\overline{h}_{\mu\nu} - a_2 R^{(1)}\gamma_{\mu\nu}\right) - \partial_\mu\partial_\nu \left(-4a_2 R\right)\right] \\ \nonumber
 = {} & 2 a_2\partial_\mu\partial_\nu R^{(1)} + \recip{6} R^{(1)}\gamma_{\mu\nu},
\end{align}
Note that the d'Almebertian is now $\Box^{(\mathrm{B})} = \gamma^{\mu\nu}\partial_\mu\partial_\nu$. The next term is
\begin{align}
{R^{(2)}}_{\mu\nu} = {} & \partial_\rho {{\Gamma^{(2)}}^\rho}_{\mu\nu} - \partial_\nu {{\Gamma^{(2)}}^\rho}_{\mu\rho} + {{\Gamma^{(1)}}^\rho}_{\mu\nu}{{\Gamma^{(1)}}^\sigma}_{\rho\sigma} - {{\Gamma^{(1)}}^\rho}_{\mu\sigma}{{\Gamma^{(1)}}^\sigma}_{\rho\nu} \nonumber \\
 = {} & \frac{1}{2}\left[\frac{1}{2}\partial_\mu h^{\rho\sigma}\partial_\nu h_{\rho\sigma} + h^{\sigma\rho}\left(\partial_\mu\partial_\nu h_{\rho\sigma} + \partial_\rho\partial_\sigma h_\mu\nu - \partial_\nu\partial_\rho h_{\mu\sigma} - \partial_\sigma\partial_\mu h_{\rho\nu}\right) \right.  \nonumber \\
 & - \left.  \left(\partial_\sigma h^{\sigma\rho} - \frac{1}{2}\partial^\rho h\right)\left(\partial_\mu h_{\rho\mu} + \partial_\nu h_{\mu\rho} - \partial_\rho h_{\mu\nu}\right) + \partial^\rho h^\sigma_\nu\left(\partial_\rho h_{\sigma\mu} - \partial_\sigma h_{\rho\mu}\right) \right] \nonumber \\
 = {} & \frac{1}{2}\left\{\recip{2}\partial_\mu\overline{h}_{\sigma\rho}\partial_\nu\overline{h}^{\sigma\rho} + \overline{h}^{\sigma\rho}\left[\partial_\mu\partial_\nu\overline{h}_{\sigma\rho} + \partial_\sigma\partial_\rho\left(\overline{h}_{\mu\nu} - a_2 R^{(1)}\gamma_{\mu\nu}\right) \right.\right. \nonumber \\
 & - \left.\left. \partial_\nu\partial_\rho\left(\overline{h}_{\sigma\mu} - a_2 R^{(1)} \gamma_{\sigma\mu}\right) - \partial_\mu\partial_\rho\left(\overline{h}_{\sigma\nu} - a_2 R^{(1)} \gamma_{\sigma\nu}\right)\right] \right. \nonumber \\
 & + \left. \partial^\rho\overline{h}^\sigma_\nu\left(\partial_\rho\overline{h}_{\sigma\mu} - \partial_\sigma\overline{h}_{\rho\mu}\right) - a_2 \partial^\sigma R^{(1)}\partial_\sigma\overline{h}_{\mu\mu} + {a_2}^2 \left[2R^{(1)}\partial_\mu\partial_\nu R^{(1)} \right.\right. \nonumber \\
 & + \left.\left. 3\partial_\mu R^{(1)}\partial_\nu R^{(1)} + R^{(1)} \Box^{(\mathrm{B})} R^{(1)} \gamma_{\mu\nu}\right]\right\}.
\end{align}

To find the Ricci scalar we must contract the Ricci tensor, but we must decide which metric to use to do this. It is tempting to use the background metric, as we used this for raising the indices on $h_{\mu\nu}$, however this was just a notational convenience. The physical metric is the full metric, so we must use this to form $R$. Remembering that we are only considering terms to $\order{\epsilon^2}$, this gives
\begin{align}
R^{(\mathrm{B})} = {} & \gamma^{\mu\nu} {R^{(\mathrm{B})}}_{\mu\nu} \\
R^{(1)} = {} & \gamma^{\mu\nu} {R^{(1)}}_{\mu\nu} \\
R^{(2)} = {} & \gamma^{\mu\nu} {R^{(2)}}_{\mu\nu} - h^{\mu\nu} {R^{(1)}}_{\mu\nu} \nonumber \\
 = {} & \gamma^{\mu\nu} {R^{(2)}}_{\mu\nu} - \overline{h}^{\mu\nu} {R^{(1)}}_{\mu\nu} + a_2 {R^{(1)}}^2 \nonumber \\
 = {} & \frac{3}{4}\partial_\mu\overline{h}_{\sigma\rho}\partial^\mu\overline{h}^{\sigma\rho} - \recip{2} \partial^\rho\overline{h}^{\sigma\mu}\partial_\sigma\overline{h}_{\rho\mu} - 2a_2 \overline{h}^{\mu\nu}\partial_\mu\partial_\nu R^{(1)} \nonumber \\
 & + {} a_2 {R^{(1)}}^2 + \frac{3a_2}{2}\partial_\mu R^{(1)} \partial^\mu R^{(1)}.
\end{align}
Using these
\begin{align}
f^{(\mathrm{B})} = {} & R^{(\mathrm{B})} \\
f^{(1)} = {} & R^{(1)} \\
f^{(2)} = {} & R^{(2)} + \frac{a_2}{2}{R^{(1)}}^2,
\end{align}
and
\begin{align}
f'^{(\mathrm{B})} = {} & a_2 R^{(\mathrm{B})} \\
f'^{(0)} = {} & 1 \\
f'^{(1)} = {} & a_2 R^{(1)} \\
f'^{(2)} = {} & a_2 R^{(2)} + \frac{a_3}{2}{R^{(1)}}^2.
\end{align}
We list a zeroth order term here for clarity.

Using all of these
\begin{align}
{\mathcal{G}^{(2)}}_{\mu\nu} = {} & {R^{(2)}}_{\mu\nu} + f'^{(1)}{R^{(1)}}_{\mu\nu} - \partial_\mu\partial_\nu f'^{(2)} + {{\Gamma^{(1)}}^\rho}_{\nu\mu}\partial_\rho f'^{(1)} + \gamma_{\mu\nu}\gamma^{\rho\sigma}\partial_\rho\partial_\sigma f'^{(2)} \nonumber \\
 & - {} \gamma_{\mu\nu}\gamma^{\rho\sigma}{{\Gamma^{(1)}}^\lambda}_{\sigma\rho}\partial_\lambda f'^{(1)} - \gamma_{\mu\nu}h^{\rho\sigma}\partial_\rho\partial_\sigma f'^{(1)} + h_{\mu\nu}\gamma^{\rho\sigma}\partial_\rho\partial_\sigma f'^{(1)} \nonumber \\
 & - {} \recip{2}f^{(2)}\gamma_{\mu\nu} - \recip{2}f^{(1)}h_{\mu\nu} \nonumber \\
 = {} & {R^{(2)}}_{\mu\nu} + a_2R^{(1)}{R^{(1)}}_{\mu\nu} - \partial_\mu\partial_\nu \left(a_2R^{(2)} + \frac{a_3}{2}{R^{(1)}}^2\right) + {{\Gamma^{(1)}}^\rho}_{\nu\mu}\partial_\rho \left(a_2R^{(1)}\right) \nonumber \\
 & + {} \gamma_{\mu\nu}\Box^{(\mathrm{B})} \left(a_2R^{(2)} + \frac{a_3}{2}{R^{(1)}}^2\right) - \gamma_{\mu\nu}\gamma^{\rho\sigma}{{\Gamma^{(1)}}^\lambda}_{\sigma\rho}\partial_\lambda\left(a_2R^{(1)}\right) \nonumber \\
 & - {} \gamma_{\mu\nu}h^{\rho\sigma}\partial_\rho\partial_\sigma\left(a_2R^{(1)}\right) + h_{\mu\nu}\Box^{(\mathrm{B})}\left(a_2R^{(1)}\right) - \recip{2}\left(R^{(2)} + \frac{a_2}{2}{R^{(1)}}^2\right)\gamma_{\mu\nu} \nonumber \\
 & - {} \recip{2}R^{(1)}h_{\mu\nu} \nonumber \\
 = {} & {R^{(2)}}_{\mu\nu} + a_2\left(\gamma_{\mu\nu}\Box^{(\mathrm{B})} - \partial_\mu\partial_\nu\right)R^{(2)} - \recip{2}R^{(2)}\gamma_{\mu\nu} + a_3\left(\gamma_{\mu\nu}\Box^{(\mathrm{B})} - \partial_\mu\partial_\nu\right){R^{(1)}}^2 \nonumber \\
 & - {} \recip{6}\overline{h}_{\mu\nu}R^{(1)} - a_2\gamma_{\mu\nu}\overline{h}^{\sigma\rho}\partial_\sigma\partial_\rho R^{(1)} + \frac{a_2}{2} \partial^\rho R^{(1)} \left(\partial_\mu\overline{h}_{\rho\nu} + \partial_\nu\overline{h}_{\rho\mu} - \partial_\rho\overline{h}_{\mu\nu}\right) \nonumber \\
 & + {} a_2\left(R^{(1)}{R^{(1)}}_{\mu\nu} + \recip{4}{R^{(1)}}^2\gamma_{\mu\nu}\right) - {a_2}^2\left(\partial_\mu R^{(1)}\partial_\nu R^{(1)} \right. \nonumber \\
 & + \left. \recip{2} \gamma_{\mu\nu}\partial^\rho R^{(1)}\partial_\rho R^{(1)}\right)
\end{align}
It is simplest to split this up for the purposes of averaging. Since we average over over all directions at each point gradients average to zero
\begin{equation}
\left\langle\partial_\mu V\right\rangle = 0.
\end{equation}
As a corollary of this we have the relation
\begin{equation}
\left\langle U\partial_\mu V\right\rangle = -\left\langle \partial_\mu U V\right\rangle.
\end{equation}
Repeated application of this, together with our gauge condition, \eqnref{Lorenz}, and wave equations, \eqref{eq:Box_R} and \eqref{eq:Box_hmunu}, allows us to eliminate many terms. Considering terms that do not trivially average to zero
\begin{equation}
\left\langle {R^{(2)}}_{\mu\nu} \right\rangle = \left\langle -\recip{4} \partial_\mu\overline{h}_{\sigma\rho}\partial^\mu\overline{h}^{\rho\sigma} + \frac{{a_2}^2}{2}\partial_\mu R^{(1)}\partial_\nu R^{(1)} + \frac{a_2}{6}\gamma_{\mu\nu}R^{(1)} \right\rangle;
\end{equation}
\begin{equation}
\left\langle R^{(2)} \right\rangle = \left\langle \frac{3a_2}{2}{R^{(1)}}^2 \right\rangle;
\end{equation}
\begin{equation}
\left\langle \overline{h}_{\mu\nu}R^{(1)} \right\rangle = 0;
\end{equation}
\begin{equation}
\left\langle R^{(1)}{R^{(1)}}_{\mu\nu} \right\rangle = \left\langle a_2 R^{(1)} \partial_\mu\partial_\nu R^{(1)} + \recip{6}\gamma_{\mu\nu}{R^{(1)}}^2\right\rangle.
\end{equation}
Combining these gives
\begin{equation}
\left\langle {\mathcal{G}^{(2)}}_{\mu\nu}\right\rangle = \left\langle -\recip{4} \partial_\mu\overline{h}_{\sigma\rho}\partial^\mu\overline{h}^{\rho\sigma} - \frac{3{a_2}^2}{2}\partial_\mu R^{(1)}\partial_\nu R^{(1)} \right\rangle.
\end{equation}
Thus we obtain the result
\begin{equation}
t_{\mu\nu} = \frac{c^4}{32\pi G}\left\langle \partial_\mu\overline{h}_{\sigma\rho}\partial^\mu\overline{h}^{\rho\sigma} + 6{a_2}^2\partial_\mu R^{(1)}\partial_\nu R^{(1)} \right\rangle.
\end{equation}
In the limit of $a_2 \rightarrow 0$ we obtain the standard GR result as required. Note that the GR result is also recovered if $R^{(1)} = 0$, as would be the case if the Ricci mode was not excited, for example if the frequency was below the cut off frequency $m$. Rewriting the pseudotensor in terms of metric perturbation $h_{\mu\nu}$, using \eqnref{gauge}, we obtain
\begin{equation}
t_{\mu\nu} = \frac{c^4}{32\pi G}\left\langle \partial_\mu h_{\sigma\rho}\partial^\mu h^{\rho\sigma} + \recip{8}\partial_\mu h \partial_\nu h \right\rangle.
\end{equation}

\section{Radiation With A Source}

Having consider radiation in a vacuum, we now move on to consider the case with a source term. We want a first order perturbation from our background metric so that the linearised field equation is
\begin{equation}
{\mathcal{G}^{(1)}}_{\mu\nu} = \frac{8\pi G}{c^4}T_{\mu\nu}.
\end{equation}
Again assuming a Minkowski background, our first order field equations are
\begin{align}
{\mathcal{G}^{(1)}}_{\mu\nu} = {} & {R^{(1)}}_{\mu\nu} - \partial_\mu\partial_\nu(a_2 R^{(1)}) + \eta_{\mu\nu}\Box(a_2 R^{(1)}) - \frac{R^{(1)}}{2}\eta_{\mu\nu} \\
\mathcal{G}^{(1)} = {} & 3a_2 \Box R^{(1)} - R^{(1)}
\end{align}
where $\mathcal{G}^{(1)} = \eta^{\mu\nu}{\mathcal{G}^{(1)}}_{\mu\nu}$. Will we again use our ansatz $\overline{h}_{\mu\nu}$ from \eqnref{Ansatz}, with $b = -1$. The Ricci tensor is
\begin{equation}
{R^{(1)}}_{\mu\nu} = \frac{1}{2}\left[2 a_2 \partial_\mu\partial_\nu R^{(1)} - \Box\left(\overline{h}_{\mu\nu} -\frac{\overline{h}}{2}\eta_{\mu\nu}\right) + \frac{a_2}{2}\Box R^{(1)}\eta_{\mu\nu}\right].
\end{equation}
Taking the traces gives
\begin{equation}
R^{(1)} = 3a_2\Box R^{(1)} + \recip{2}\Box\overline{h} 
\end{equation}
just as in \eqnref{Ricci_Box_h}. Comparing this with the trace of the field equation yields
\begin{equation}
-\recip{2}\Box\overline{h} = \mathcal{G}^{(1)}
\end{equation}
Using this with our expression for the Ricci tensor in the field equation gives
\begin{align}
{\mathcal{G}^{(1)}}_{\mu\nu} = {} & a_2\partial_\mu\partial_\nu R^{(1)} - \recip{2}\Box\overline{h}_{\mu\nu} + \recip{2}\Box\left(\frac{\overline{h}}{2} + a_2 R^{(1)}\right)\eta_{\mu\nu} - a_2\partial_\mu\partial_\nu R^{(1)}  \nonumber \\
 & + {} a_2\Box R^{(1)} \eta_{\mu\nu} - \recip{2}R^{(1)}\eta_{\mu\nu} \nonumber \\
 = {} & - \recip{2}\Box\overline{h}_{\mu\nu}.
\end{align}
We see that our field equations are consistent.

\chapter{Parabolic Encounters Of A Supermassive Black Hole}

\chapter{Further Work}

\newpage
\bibliographystyle{../physics}
\bibliography{../library}

\end{document}