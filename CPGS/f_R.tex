\chapter{Gravitational Radiation In $f(R)$ Theory}

\section{Introduction To $f(R)$ Theory}

General relativity (GR) is a well tested theory of gravity\cite{Will2006}; however it is still interesting to explore alternative theories. This may be motivated by the need to explain dark matter and dark matter in cosmology, trying to formulate a quantizable theory of gravity or simple curiosity regarding the uniqueness of GR. One of the simplest extensions to standard GR is the class of $f(R)$ theories\cite{Sotiriou2010, DeFelice2010}.

\subsection{The Action \& Field Equations}\label{sec:Action}

General relativity may be derived from the Einstein-Hilbert action\cite{Misner1973, Landau1975}
\begin{equation}
S\sub{EH}[g] = \frac{c^4}{16\pi G}\intd{}{}{R\sqrt{-g}}{^4x}.
\end{equation}
In $f(R)$ theory we make a simple modification of the action to include an arbitrary function of the Ricci scalar $R$ such that\cite{Buchdahl1970}
\begin{equation}
S[g] = \frac{c^4}{16\pi G}\intd{}{}{f(R)\sqrt{-g}}{^4x}.
\end{equation}
Including the function $f(R)$ gives extra freedom in defining the behaviour of gravity; while this action may not encode the true theory of gravity it may at least contain sufficient information to act as an effective field theory, correctly describing phenomological behaviour\cite{Park2010}. We will assume that $f(R)$ is analytic about $R = 0$ so that it may be expressed as a power series\cite{Buchdahl1970, Psaltis2008}
\begin{equation}
f(R) = a_0 + a_1 R + \frac{a_2}{2!}R^2 + \frac{a_3}{3!}R^3 + \ldots
\end{equation}
Note that since the dimensions of $f(R)$ must be the same as of $R$, $[a_n] = [R]^{(1-n)}$. To link to GR we will set $a_1 = 1$; any rescaling may be absorbed into the definition of $G$.

The field equations are obtained by a variational principle; there are a number of choices for achieving this. To derive the Einstein field equations from the Einstein-Hilbert action one may use the standard metric variation or the Palatini variation\cite{Misner1973}. Both approaches may be used for $f(R)$, however they now yield different results\cite{Sotiriou2010, DeFelice2010}. Following the metric (or second order) formalism, one varies the action with respect to the metric $g^{\mu\nu}$, the resulting field equations being those for metric $f(R)$ gravity. Following the Palantini (or first order) formalism one varies the action with respect both to the metric $g^{\mu\nu}$ and to the connection ${\Gamma^\rho}_{\mu\nu}$, which are treated as independent quantities: the connection is not the Levi-Cevita metric connection.\footnote{Imposing that the metric and Palantini formalisms produce the same field equations, assuming an action that only depends on the metric and Reimann tensor, results in Lovelock gravity\cite{Exirifard2008}. Lovelock gravities require the field equations to be divergence free and no more than second order; in four dimensions the only possible Lovelock gravity is GR with a potentially non-zero cosmological constant\cite{Lovelock1970, Lovelock1971, Lovelock1972}.}

Finally, there is a third version of $f(R)$ gravity: metric-affine $f(R)$ gravity\cite{Sotiriou2007, Sotiriou2007b}. This goes beyond the Palantini formalism by supposing that the matter action is dependent on the variational independent connection. Parallel transport and the covariant derivative are divorced from the metric. This theory has its attractions: it allows for a natural introduction or torsion. However, it is not a metric theory of gravity and so cannot satisfy all the postulates of the Einstein equivalence principle\cite{Will2006}: a free particle does not necessarily follow a geodesic and so the effects of gravity may not be locally removed\cite{Exirifard2008}. The implications of this have not been fully explored, but for this reason we shall not consider the theory further.

We shall restrict our attention to consider only metric $f(R)$ gravity. The metric formalism is preferred as the Palantini formalism has undesirable properties: static spherically symmetric objects described by a polytropic equation of state are subject to a curvature singularity\cite{Barausse2008b, Barausse2008a}. Varying the action with respect to the metric $g^{\mu\nu}$ produces
\begin{equation}
\delta S = \frac{c^4}{16\pi G}\intd{}{}{\left\{f'(R)\sqrt{-g}\left[R_{\mu\nu} - \nabla_\mu\nabla_\nu\ + g_{\mu\nu}\Box\right] - f(R)\frac{1}{2\sqrt{-g}}gg_{\mu\nu}\right\}\delta g^{\mu\nu}}{^4x},
\end{equation}
where $\Box = g^{\mu\nu}\nabla_\mu\nabla_\nu$ is the d'Alembertian and a prime denotes differentiation with respect to $R$. Proceeding from here requires certain assumptions regarding surface terms. In the case of the Einstein-Hilbert action the surface terms gather into a total derivative. It is possible to subtract this from the action to obtain a well-defined variational quantity\cite{York1972, Gibbons1977}. However, this is not the case for general $f(R)$\cite{Madsen1989}. It is argued that since the action includes higher order derivatives of the metric we are at liberty to fix more degrees of freedom at the boundary, in so doing eliminating the importance of the surface terms\cite{Dyer2009a, Sotiriou2010}. There is no well described prescription for this so we proceed directly to the field equations.

The vacuum field equation is
\begin{equation}
f'R_{\mu\nu} - \nabla_\mu\nabla_\nu f' + g_{\mu\nu}\Box f' - \frac{f}{2}g_{\mu\nu} = 0.
\label{eq:Field_eq}
\end{equation}
For standard GR, when $f(R) = R$, this reduces to the familiar
\begin{equation}
R_{\mu\nu} - \frac{R}{2}g_{\mu\nu} = 0.
\end{equation}
Taking the trace of our field equation gives
\begin{equation}
f'R + 3\Box f' - 2f = 0.
\label{eq:Trace_eq}
\end{equation}
Note if we consider a uniform flat spacetime $R = 0$, this equation gives
\begin{equation}
a_0 = 0.
\label{eq:a_0}
\end{equation}
In analogy to the Einstein tensor, we shall define
\begin{equation}
\mathcal{G}_{\mu\nu} = f'R_{\mu\nu} - \nabla_\mu\nabla_\nu f' + g_{\mu\nu}\Box f' - \frac{f}{2}g_{\mu\nu},
\label{eq:G_tensor}
\end{equation}
so that in a vacuum
\begin{equation}
\mathcal{G}_{\mu\nu} = 0.
\end{equation}

\subsection{Conservation Of Energy-Momentum}

If we introduce matter with a stress-energy tensor $T_{\mu\nu}$, the field equation becomes
\begin{equation}
\mathcal{G}_{\mu\nu} = \frac{8\pi G}{c^4}T_{\mu\nu}.
\end{equation}
If we act upon this with the covariant derivative we obtain
\begin{align}
\frac{8\pi G}{c^4}\nabla^\mu T_{\mu\nu} = {} & \nabla^\mu\mathcal{G}_{\mu\nu} \nonumber \\
= {} & R_{\mu\nu}\nabla^\mu f' + f'\nabla^\mu\left(R_{\mu\nu} - \recip{2}R g_{\mu\nu}\right) - \left(\Box\nabla_\nu - \nabla_\nu\Box\right)f'.
\end{align}
The second term contains the covariant derivative of the Einstein tensor and so is zero. After some manipulation the final term can be shown to be
\begin{align}
\left(\Box\nabla_\nu - \nabla_\nu\Box\right)f' = {} & g^{\mu\sigma}\left[\nabla_\mu\nabla_\sigma\nabla_\nu - \nabla_\nu\nabla_\mu\nabla_\sigma\right]f' \nonumber \\
 = {} & R_{\tau\nu}\nabla^\tau f',
\end{align}
which is a useful geometric identity\cite{Koivisto2006a}. Using this we find that
\begin{align}
\frac{8\pi G}{c^4}\nabla^\mu T_{\mu\nu} = {} & R_{\mu\nu}\nabla^\mu f' - R_{\mu\nu}\nabla^\mu f' \nonumber \\
 = {} & 0.
\end{align}
Consequently we see that energy-momentum is a conserved quantity in the same way as in GR, as may be expected from the symmetries of the action.

\section{Linearized Theory}\label{sec:Lin}

We will now consider the case that the metric is perturbed slightly from flat Minkowski such that
\begin{equation}
g_{\mu\nu} = \eta_{\mu\nu} + h_{\mu\nu};
\end{equation}
where, more formally, we mean that $h_{\mu\nu} = \varepsilon H_{\mu\nu}$ for small parameter $\varepsilon$.\footnote{It is because we wish to perturb about flat spacetime that we have required $f(R)$ to be analytic about $R = 0$.} We will consider terms only to $\order{\varepsilon}$. Thus, the inverse metric is
\begin{equation}
g^{\mu\nu} = \eta^{\mu\nu} - h^{\mu\nu},
\end{equation}
where we have used the Minkowski metric to raise the indices on the right side, effectively defining
\begin{equation}
h^{\mu\nu} = \eta^{\mu\sigma}\eta^{\nu\rho}h_{\sigma\rho}.
\end{equation}
Similarly, the trace $h$ is given by
\begin{equation}
h = \eta^{\mu\nu}h_{\mu\nu}.
\end{equation}
This means that all quantities denoted by ``$h$'' are strictly $\order{\varepsilon}$. We will have to be careful later on to distinguish between quantities where the Minkowski metric has been used to raise indices and those where the full metric has been used.

The linearized ($\order{\varepsilon}$) connection coefficient is
\begin{equation}
{{\Gamma^{(1)}}^\rho}_{\mu\nu} = \frac{1}{2}\eta^{\rho\lambda}(\partial_\mu h_{\lambda\nu} + \partial_\nu h_{\lambda\mu} - \partial_\lambda h_{\mu\nu}).
\label{eq:Lin_Gamma}
\end{equation}
The covariant derivative of any perturbed quantity will be the same as the partial derivative to first order. The Riemann tensor is
\begin{equation}
{{R^{(1)}}^\lambda}_{\mu\nu\rho} = \frac{1}{2}(\partial_\mu\partial_\nu h^\lambda_\rho + \partial^\lambda\partial_\rho h_{\mu\nu} - \partial_\mu\partial_\rho h^\lambda_\nu - \partial^\lambda\partial_\nu h_{\mu\rho}),
\label{eq:Lin_Riemann}
\end{equation}
where we have raised the index on the differential operator with the background Minkowski metric. Contracting gives the Ricci tensor
\begin{equation}
{R^{(1)}}_{\mu\nu} = \frac{1}{2}(\partial_\mu\partial_\rho h^\rho_\nu + \partial_\nu\partial_\rho h^\rho_\mu -\Box h_{\mu\nu} - \partial_\mu\partial_\nu h),
\label{eq:Ricci}
\end{equation}
where the d'Alembertian operator is $\Box = \eta^{\mu\nu}\partial_\mu\partial_\nu$. Contracting this with $\eta^{\mu\nu}$ we find the first order Ricci scalar
\begin{equation}
R^{(1)} = \partial_\mu\partial_\rho h^{\rho\mu} - \Box h.
\label{eq:Scalar}
\end{equation}

Since $R^{(1)}$ is $\order{\varepsilon}$ we may write $f(R)$ as a Maclaurin series to first order such that
\begin{align}
f(R) = {} & a_0 + R^{(1)}\\
f'(R) = {} & 1 + a_2 R^{(1)}.
\end{align}
As we are perturbing from a flat Minkowski background where the Ricci scalar vanishes, we may use \eqnref{a_0} to set $a_0 = 0$. Inserting these into \eqnref{G_tensor} and retaining terms to first order we obtain
\begin{equation}
{\mathcal{G}^{(1)}}_{\mu\nu} = {R^{(1)}}_{\mu\nu} - \partial_\mu\partial_\nu(a_2 R^{(1)}) + \eta_{\mu\nu}\Box(a_2 R^{(1)}) - \frac{R^{(1)}}{2}\eta_{\mu\nu}.
\label{eq:Field}
\end{equation}
We see that we need to find a relation between $R^{(1)}$ and its derivatives. Let us consider the linearized trace equation, from \eqnref{Trace_eq}
\begin{align}
\mathcal{G}^{(1)} = {} & R^{(1)} + 3 \Box(a_2 R^{(1)}) - 2 R^{(1)} \nonumber \\
\mathcal{G}^{(1)} = {} & 3a_2 \Box R^{(1)} - R^{(1)},
\label{eq:Box_R}
\end{align}
where $\mathcal{G}^{(1)} = \eta^{\mu\nu}{\mathcal{G}^{(1)}}_{\mu\nu}$. This is the massive inhomogeneous Klein-Gordon equation. Setting $\mathcal{G} = 0$ as for a vacuum we obtain the standard Klein-Gordon equation
\begin{equation}
\Box R^{(1)} + \kappa^2 R^{(1)} = 0,
\end{equation}
if we define inverse length scale
\begin{equation}
\kappa^2 = -\recip{3a_2}.
\end{equation}
For a physically meaningful solution we require $\kappa^2 > 0$, thus we constrain $f(R)$ such that $a_2 < 0$\cite{Schmidt1986, Teyssandier1990, Olmo2005c, Corda2007}. From the inverse length scale $\kappa$ we may define a reduced Compton wavelength
\begin{equation}
\lambdabar_R = \recip{\kappa},
\end{equation}
and mass
\begin{equation}
m_R = \frac{\hbar\kappa}{c}
\end{equation}
associated with this scalar mode.

The next step is to substitute in $h_{\mu\nu}$ to try to find wave solutions. We hope to find a quantity $\overline{h}_{\mu\nu}$ that will satisfy a wave equation, related to $h_{\mu\nu}$ by
\begin{equation}
\overline{h}_{\mu\nu} = h_{\mu\nu} + A_{\mu\nu}.
\end{equation}
In GR we use the trace-reversed form where $A_{\mu\nu} = -(h/2)\eta_{\mu\nu}$. This will not suffice here as we have additional terms, but let us look for a similar solution
\begin{equation}
\overline{h}_{\mu\nu} = h_{\mu\nu} - \frac{h}{2}\eta_{\mu\nu} + B_{\mu\nu}.
\end{equation}
The only rank two tensors in our theory are: $h_{\mu\nu}$, $\eta_{\mu\nu}$, ${R^{(1)}}_{\mu\nu}$, and $\partial_\mu\partial_\nu$; $h_{\mu\nu}$ has been used already, and we wish to eliminate ${R^{(1)}}_{\mu\nu}$, so we will try the simpler option based around $\eta_{\mu\nu}$. We want $B_{\mu\nu}$ to be $\order{\varepsilon}$. There are three scalar quantities that satisfy this: $h$, $R^{(1)}$ and $\Box R^{(1)}$; $h$ is used already and $\Box R^{(1)}$ is related to $^{(1)}R$ by \eqnref{Box_R}. Therefore, we may construct an ansatz
\begin{equation}
\overline{h}_{\mu\nu} = h_{\mu\nu} + \left(b a_2 R^{(1)} - \frac{h}{2}\right)\eta_{\mu\nu}
\label{eq:Ansatz}
\end{equation}
where $a_2$ has been included to ensure dimensional consistency and $b$ is a dimensionless number. Contracting with the background metric yields
\begin{equation}
\overline{h} = 4b a_2 R^{(1)} - h,
\label{eq:h_trace}
\end{equation}
so we may eliminate $h$ in our definition of $\overline{h}_{\mu\nu}$ to give
\begin{equation}
h_{\mu\nu} = \overline{h}_{\mu\nu} + \left(b a_2 R^{(1)} -\frac{\overline{h}}{2}\right)\eta_{\mu\nu}.
\end{equation}
We will assume a Lorenz, or de Donder, gauge choice such that
\begin{equation}
\nabla^\mu \overline{h}_{\mu\nu} = 0,
\label{eq:Lorenz}
\end{equation}
to first order this gives
\begin{equation}
\partial^\mu \overline{h}_{\mu\nu} = 0.
\end{equation}
Subject to this, the Ricci tensor is, from \eqnref{Ricci},
\begin{equation}
{R^{(1)}}_{\mu\nu} = -\frac{1}{2}\left\{2b a_2 \partial_\mu\partial_\nu R^{(1)} + \Box\left(\overline{h}_{\mu\nu} -\frac{\overline{h}}{2}\eta_{\mu\nu}\right) + \frac{b}{3}R^{(1)}\eta_{\mu\nu}\right\}.
\end{equation}
Using this together with \eqnref{Box_R} in our field equation \eqref{eq:Field} gives
\begin{equation}
-\frac{1}{2}\Box\left(\overline{h}_{\mu\nu} - \frac{\overline{h}}{2}\right) - (b + 1)\left(a_2\partial_\mu\partial_\nu R + \frac{R}{6}\eta_{\mu\nu}\right) = {\mathcal{G}^{(1)}}_{\mu\nu} - \mathcal{G}^{(1)}\eta_{\mu\nu}.
\label{eq:b_Field}
\end{equation}
If we pick $b = -1$, then the second term vanishes, thus we will set\cite{Corda2007, Capozziello2008}
\begin{align}
\overline{h}_{\mu\nu} = {} & h_{\mu\nu} - \left(a_2 R^{(1)} + \frac{h}{2}\right)\eta_{\mu\nu}\\
h_{\mu\nu} = {} & \overline{h}_{\mu\nu} - \left(a_2 R^{(1)} -\frac{\overline{h}}{2}\right)\eta_{\mu\nu}.
\label{eq:h_metric}
\end{align}
Let us now consider the Ricci scalar in this case, then from \eqnref{Scalar}
\begin{align}
R^{(1)} = {} & \Box \left(a_2 R^{(1)} -\frac{\overline{h}}{2}\right) - \Box (-4 a_2 R^{(1)} - \overline{h}) \nonumber \\
 = {} & 3a_2 \Box R^{(1)} + \frac{1}{2}\Box \overline{h}.
\label{eq:Ricci_Box_h}
\end{align}
For consistency with \eqnref{Box_R}, we see that
\begin{equation}
-\recip{2}\Box \overline{h} = \mathcal{G}^{(1)}.
\label{eq:Box_h}
\end{equation}
Inserting this into \eqnref{b_Field}, with $b = -1$, we see
\begin{equation}
-\recip{2}\Box \overline{h}_{\mu\nu} = {\mathcal{G}^{(1)}}_{\mu\nu};
\label{eq:Box_hmunu}
\end{equation}
we have our wave equation and it is consistent.

Should $a_2$ be sufficiently small that it may be regarded an $\order{\varepsilon}$ quantity, we recover GR to leading order within our analysis.

\section{Gravitational Radiation}

Having established two wave equations, \eqref{eq:Box_R} and \eqref{eq:Box_hmunu}, we may now investigate their solutions. We shall consider waves in a vacuum such that $\mathcal{G}_{\mu\nu} = 0$. Using a standard Fourier decomposition
\begin{align}
\overline{h}_{\mu\nu} = {} & \Re\left\{\widehat{\overline{h}}_{\mu\nu}(k_\rho) \exp\left(ik_\rho x^\rho\right)\right\},\\
R^{(1)} = {} & \Re\left\{\widehat{R}(q_\rho) \exp\left(iq_\rho x^\rho\right)\right\},
\end{align}
where $k_\mu$ and $q_\mu$ are the 4-wavevectors of the waves. From \eqnref{Box_hmunu} we know that $k_\mu$ is a null vector, so for a wave travelling along the $z$-axis
\begin{equation}
k^\mu = \frac{\omega}{c}(1, 0, 0, 1),
\end{equation}
where $\omega$ is the angular frequency. Similarly, from \eqnref{Box_R}
\begin{equation}
q^\mu = \left(\frac{\Omega}{c}, 0, 0, \sqrt{\frac{\Omega^2}{c^2} - \kappa^2}\right),
\label{eq:Ricci_q}
\end{equation}
for frequency $\Omega$. These waves do not travel at $c$, but have a group velocity
\begin{equation}
v = \frac{c\sqrt{\Omega^2 - c^2\kappa^2}}{\Omega}.
\end{equation}
Provided that $\kappa^2 > 0$, $v < c$, but we do not get propagating modes for $\Omega < \Omega_R = c\kappa$.

From the condition \eqref{eq:Lorenz} we find that $k^\mu$ is orthogonal to $\widehat{\overline{h}}_{\mu\nu}$,
\begin{equation}
k^\mu\widehat{\overline{h}}_{\mu\nu} = 0,
\end{equation}
thus in this case
\begin{equation}
\widehat{\overline{h}}_{0\nu} + \widehat{\overline{h}}_{3\nu} = 0.
\label{eq:Transverse}
\end{equation}

Let us now consider the implications of \eqnref{Box_h} using equations \eqref{eq:h_trace} and \eqref{eq:Box_R},
\begin{align}
\Box\left(4a_2R^{(1)} + h\right) = {} & 0 \nonumber \\
\Box h = {} & -\frac{4}{3}R^{(1)}.
\end{align}
For non-zero $R^{(1)}$ (as required for the Ricci mode) there is no way we can make such that the trace $h$ will vanish\cite{Corda2007, Capozziello2008}. This is distinct from the case in GR. It is possible, however, to make a gauge choice such that the trace $\overline{h}$ will vanish. Consider a gauge transformation generated by $\xi_\mu$ which satisfies $\Box \xi_\mu = 0$, and so has a Fourier decomposition
\begin{equation}
\xi_\mu = \widehat{\xi}_\mu \exp\left(ik_\rho x^\rho\right).
\end{equation}
We see that a transformation
\begin{equation}
\overline{h}_{\mu\nu} \rightarrow \overline{h}_{\mu\nu} + \partial_\mu\xi_\nu + \partial_\nu\xi_\mu - \eta_{\mu\nu}\partial^\rho\xi_\rho,
\end{equation}
would ensure both conditions \eqref{eq:Lorenz} and \eqref{eq:Box_hmunu} are satisfied\cite{Misner1973}. Under such a transformation
\begin{equation}
\widehat{\overline{h}}_{\mu\nu} \rightarrow \widehat{\overline{h}}_{\mu\nu} + i\left(k_\mu\widehat{\xi}_\nu + k_\nu\widehat{\xi}_\mu - \eta_{\mu\nu}k^\rho\widehat{\xi}_\rho\right).
\end{equation}
We may therefore impose four further constraints (one for each $\widehat{\xi}_\mu$) upon $\widehat{\overline{h}}_{\mu\nu}$, and we may take these to be
\begin{equation}
\widehat{\overline{h}}_{0\nu} = 0, \quad \widehat{\overline{h}} = 0.
\end{equation}
This may appear to be five constraints, however we have already imposed \eqref{eq:Transverse}, and so setting $\widehat{\overline{h}}_{00} = 0$ automatically implies $\widehat{\overline{h}}_{03} = 0$. In this gauge we have
\begin{align}
h_{\mu\nu} = {} & \overline{h}_{\mu\nu} - a_2 R^{(1)}\eta_{\mu\nu},\\
h = {} & -4a_2R^{(1)}.
\label{eq:gauge}
\end{align}
We see that $\overline{h}_{\mu\nu}$ behaves just as its counterpart in GR so we may define
\begin{equation}
\left[\widehat{\overline{h}}_{\mu\nu}\right] =
\begin{bmatrix}
0 & 0 & 0 & 0\\
0 & h_+ & h_\times & 0\\
0 & h_\times & -h_+ & 0\\
0 & 0 & 0 & 0
\end{bmatrix},
\end{equation}
where $h_+$ and $h_\times$ are appropriate constants representing the amplitudes of the two transverse polarizations of gravitational radiation.

It is important that our solutions reduce to those of GR in the event that $f(R) = R$. In this linearized approach this corresponds to $a_2 \rightarrow 0$, $\kappa^2 \rightarrow \infty$. We see from \eqnref{Ricci_q} that in this limit it would take an infinite frequency to excite a propagating Ricci mode, and evanescent waves would decay away infinitely quickly. Therefore there would not be any detectable Ricci modes and we would only observe the two polarizations found in the analysis of GR. Additionally $\overline{h}_{\mu\nu}$ would simplify to its usual trace-reversed form.

\section{Energy-momentum Tensor}\label{sec:EM_tensor}

We expect that the gravitational field would carry energy-momentum. Unfortunately it is difficult to define a proper energy-momentum tensor for a gravitational field: as a consequence of the equivalence principle it is possible to transform to a freely falling frame, eliminating the gravitational field and any associated energy density for a given event, although we may still define curvature in the neighbourhood of this point\cite{Misner1973, Hobson2006}. We will do nothing revolutionary here, but shall follow the approach of Isaacson\cite{Isaacson1968, Isaacson1968a}. The full field equation, \eqnref{Field_eq}, has no energy-momentum tensor for the gravitational field on the right-hand side. However, by expanding beyond the linear terms we may find a suitable energy-momentum pseudotensor for gravitational radiation. We may expand $\mathcal{G}_{\mu\nu}$ in orders of $h_{\mu\nu}$
\begin{equation}
\mathcal{G}_{\mu\nu} = {\mathcal{G}^{(\mathrm{B})}}_{\mu\nu} + {\mathcal{G}^{(1)}}_{\mu\nu} + {\mathcal{G}^{(2)}}_{\mu\nu} + \ldots
\label{eq:G_exp}
\end{equation}
We use $(\mathrm{B})$ for the background term instead of $(0)$ to avoid confusion regarding its order in $\varepsilon$. Our linearised vacuum equation would then read
\begin{equation}
{\mathcal{G}^{(1)}}_{\mu\nu} = 0.
\end{equation}
So far we have assumed that our background is flat, however we can imagine that should the gravitational radiation carry energy-momentum then this would act as a source of curvature for the background. This is a second-order effect that may be encoded, to accuracy of $\order{\varepsilon^2}$, as
\begin{equation}
{\mathcal{G}^{(\mathrm{B})}}_{\mu\nu} = -{\mathcal{G}^{(2)}}_{\mu\nu}.
\end{equation}
By shifting ${\mathcal{G}^{(2)}}_{\mu\nu}$ to the right-hand side we effectively create an energy-momentum tensor. As in GR we will average over several wavelengths, assuming that the background curvature is on a larger scale\cite{Misner1973},
\begin{equation}
{\mathcal{G}^{(\mathrm{B})}}_{\mu\nu} = -\left\langle{\mathcal{G}^{(2)}}_{\mu\nu}\right\rangle.
\end{equation}
By averaging we may probe the curvature in a macroscopic region about a given point in spacetime. This gives a gauge invariant measure of the gravitational field strength. The averaging can be thought of as smoothing out the rapidly varying ripples of the radiation, leaving only the coarse-grained component that acts as a source for the background curvature.\footnote{By averaging we do not try to localise the energy of a wave to within a wavelength; for the massive Ricci scalar mode we always consider scales greater than $\lambdabar_R$.} The energy-momentum pseudotensor for the radiation may be identified as
\begin{equation}
t_{\mu\nu} = -\frac{c^4}{8\pi G}\left\langle{\mathcal{G}^{(\mathrm{2})}}_{\mu\nu}\right\rangle.
\end{equation}

Having made this provisional identification, we must now set about carefully evaluating the various terms in \eqnref{G_exp}. We begin as in \secref{Lin} by defining a total metric
\begin{equation}
g_{\mu\nu} = \gamma_{\mu\nu} + h_{\mu\nu},
\end{equation}
where $\gamma_{\mu\nu}$ is our background metric. This is changing slightly our definition for $h_{\mu\nu}$: instead of it being the total perturbation from flat Minkowski, it is the dynamical part of the metric with which we associate radiative effects. Since we know that ${\mathcal{G}^{(\mathrm{B})}}_{\mu\nu}$ is $\order{\varepsilon^2}$, we may decompose our background metric as
\begin{equation}
\gamma_{\mu\nu} = \eta_{\mu\nu} + j_{\mu\nu},
\end{equation}
where $j_{\mu\nu}$ is $\order{\varepsilon^2}$ to ensure that ${{R^{(\mathrm{B})}}^\lambda}_{\mu\nu\rho}$ is also $\order{\varepsilon^2}$. Therefore its introduction will make no difference to the linearized theory.

We will consider terms only to $\order{\varepsilon^2}$. We identify ${{\Gamma^{(1)}}^\rho}_{\mu\nu}$ from \eqnref{Lin_Gamma} to the accuracy of our analysis. There is one small subtlety: whether we use the background metric $\gamma^{\mu\nu}$ or $\eta^{\mu\nu}$ to raise indices of $\partial_\mu$ and $h_{\mu\nu}$. Fortunately, to the accuracy considered here, it does not make a difference; however, we will consider the indices to be changed with the background metric. This is more appropriate for considering the effect of curvature on gravitational radiation. We will not distinguish between $\partial_\mu$ and ${\nabla^{(\mathrm{B})}}_\mu$, the covariant derivative for the background metric: note that to the order of accuracy considered here covariant derivatives would commute and ${\nabla^{(\mathrm{B})}}_\mu$ behaves just like $\partial_\mu$. The connection coefficient has
\begin{align}
{{\Gamma^{(1)}}^\rho}_{\mu\nu} = {} & \frac{1}{2}\gamma^{\rho\lambda}\left[\partial_\mu \left(\overline{h}_{\lambda\nu} - a_2 R^{(1)}\gamma_{\lambda\nu}\right) + \partial_\nu \left(\overline{h}_{\lambda\mu} - a_2 R^{(1)}\gamma_{\lambda\mu}\right) \right. \nonumber \\
  & - \left. \partial_\lambda \left(\overline{h}_{\mu\nu} - a_2 R^{(1)}\gamma_{\mu\nu}\right)\right],
\end{align}
and
\begin{align}
{{\Gamma^{(2)}}^\rho}_{\mu\nu} = {} & -\frac{1}{2}h^{\rho\lambda}(\partial_\mu h_{\lambda\nu} + \partial_\nu h_{\lambda\mu} - \partial_\lambda h_{\mu\nu}) \nonumber \\
 = {} & -\frac{1}{2}\left(\overline{h}^{\rho\lambda} - a_2 R^{(1)}\gamma^{\rho\lambda}\right)\left[\partial_\mu \left(\overline{h}_{\lambda\nu} - a_2 R^{(1)}\gamma_{\lambda\nu}\right) + \partial_\nu \left(\overline{h}_{\lambda\mu} - a_2 R^{(1)}\gamma_{\lambda\mu}\right) \right. \nonumber \\
 & - \left. \partial_\lambda \left(\overline{h}_{\mu\nu} - a_2 R^{(1)}\gamma_{\mu\nu}\right)\right].
\end{align}
The Riemann tensor is
\begin{equation}
{R^\lambda}_{\mu\nu\rho} = {{R^{(\mathrm{B})}}^\lambda}_{\mu\nu\rho} + {{R^{(1)}}^\lambda}_{\mu\nu\rho} + {{R^{(2)}}^\lambda}_{\mu\nu\rho} + \ldots
\end{equation}
We may use our expression from \eqnref{Lin_Riemann} for ${{R^{(1)}}^\lambda}_{\mu\nu\rho}$. Contracting gives the Ricci tensor
\begin{equation}
{R}_{\mu\nu} = {R^{(\mathrm{B})}}_{\mu\nu} + {R^{(1)}}_{\mu\nu} + {R^{(2)}}_{\mu\nu} + \ldots
\end{equation}
We may again use our linearized expression, \eqnref{Ricci}, for the first order term,
\begin{equation}
{R^{(1)}}_{\mu\nu} = 2 a_2\partial_\mu\partial_\nu R^{(1)} + \recip{6} R^{(1)}\gamma_{\mu\nu}.
\end{equation}
The next term is
\begin{align}
{R^{(2)}}_{\mu\nu} = {} & \partial_\rho {{\Gamma^{(2)}}^\rho}_{\mu\nu} - \partial_\nu {{\Gamma^{(2)}}^\rho}_{\mu\rho} + {{\Gamma^{(1)}}^\rho}_{\mu\nu}{{\Gamma^{(1)}}^\sigma}_{\rho\sigma} - {{\Gamma^{(1)}}^\rho}_{\mu\sigma}{{\Gamma^{(1)}}^\sigma}_{\rho\nu} \nonumber \\
 = {} & \frac{1}{2}\left\{\recip{2}\partial_\mu\overline{h}_{\sigma\rho}\partial_\nu\overline{h}^{\sigma\rho} + \overline{h}^{\sigma\rho}\left[\partial_\mu\partial_\nu\overline{h}_{\sigma\rho} + \partial_\sigma\partial_\rho\left(\overline{h}_{\mu\nu} - a_2 R^{(1)}\gamma_{\mu\nu}\right) \right.\right. \nonumber \\
 & - \left.\left. \partial_\nu\partial_\rho\left(\overline{h}_{\sigma\mu} - a_2 R^{(1)} \gamma_{\sigma\mu}\right) - \partial_\mu\partial_\rho\left(\overline{h}_{\sigma\nu} - a_2 R^{(1)} \gamma_{\sigma\nu}\right)\right] \right. \nonumber \\
 & + \left. \partial^\rho\overline{h}^\sigma_\nu\left(\partial_\rho\overline{h}_{\sigma\mu} - \partial_\sigma\overline{h}_{\rho\mu}\right) - a_2 \partial^\sigma R^{(1)}\partial_\sigma\overline{h}_{\mu\mu} + {a_2}^2 \left[2R^{(1)}\partial_\mu\partial_\nu R^{(1)} \right.\right. \nonumber \\
 & + \left.\left. 3\partial_\mu R^{(1)}\partial_\nu R^{(1)} + R^{(1)} \Box^{(\mathrm{B})} R^{(1)} \gamma_{\mu\nu}\right]\right\}.
\end{align}
Note that the d'Almebertian is now $\Box^{(\mathrm{B})} = \gamma^{\mu\nu}\partial_\mu\partial_\nu$.

To find the Ricci scalar we must contract the Ricci tensor, but we must decide which metric to use. It is tempting to use the background metric, as we used this for raising the indices on $h_{\mu\nu}$, however this was just a notational convenience. The physical metric is the full metric, so we must use this to form $R$. Remembering that we are only considering terms to $\order{\varepsilon^2}$, this gives
\begin{align}
R^{(\mathrm{B})} = {} & \gamma^{\mu\nu} {R^{(\mathrm{B})}}_{\mu\nu} \\
R^{(1)} = {} & \gamma^{\mu\nu} {R^{(1)}}_{\mu\nu} \\
R^{(2)} = {} & \gamma^{\mu\nu} {R^{(2)}}_{\mu\nu} - h^{\mu\nu} {R^{(1)}}_{\mu\nu} \nonumber \\
 = {} & \frac{3}{4}\partial_\mu\overline{h}_{\sigma\rho}\partial^\mu\overline{h}^{\sigma\rho} - \recip{2} \partial^\rho\overline{h}^{\sigma\mu}\partial_\sigma\overline{h}_{\rho\mu} - 2a_2 \overline{h}^{\mu\nu}\partial_\mu\partial_\nu R^{(1)} \nonumber \\
 & + {} a_2 {R^{(1)}}^2 + \frac{3a_2}{2}\partial_\mu R^{(1)} \partial^\mu R^{(1)}.
\end{align}
Using these
\begin{align}
f^{(\mathrm{B})} = {} & R^{(\mathrm{B})} \\
f^{(1)} = {} & R^{(1)} \\
f^{(2)} = {} & R^{(2)} + \frac{a_2}{2}{R^{(1)}}^2,
\end{align}
and
\begin{align}
f'^{(\mathrm{B})} = {} & a_2 R^{(\mathrm{B})} \\
f'^{(0)} = {} & 1 \\
f'^{(1)} = {} & a_2 R^{(1)} \\
f'^{(2)} = {} & a_2 R^{(2)} + \frac{a_3}{2}{R^{(1)}}^2.
\end{align}
We list a zeroth order term here for clarity.

Combining all of these
\begin{align}
{\mathcal{G}^{(2)}}_{\mu\nu} = {} & {R^{(2)}}_{\mu\nu} + f'^{(1)}{R^{(1)}}_{\mu\nu} - \partial_\mu\partial_\nu f'^{(2)} + {{\Gamma^{(1)}}^\rho}_{\nu\mu}\partial_\rho f'^{(1)} + \gamma_{\mu\nu}\gamma^{\rho\sigma}\partial_\rho\partial_\sigma f'^{(2)} \nonumber \\
 & - {} \gamma_{\mu\nu}\gamma^{\rho\sigma}{{\Gamma^{(1)}}^\lambda}_{\sigma\rho}\partial_\lambda f'^{(1)} - \gamma_{\mu\nu}h^{\rho\sigma}\partial_\rho\partial_\sigma f'^{(1)} + h_{\mu\nu}\gamma^{\rho\sigma}\partial_\rho\partial_\sigma f'^{(1)} \nonumber \\
 & - {} \recip{2}f^{(2)}\gamma_{\mu\nu} - \recip{2}f^{(1)}h_{\mu\nu} \nonumber \\
 = {} & {R^{(2)}}_{\mu\nu} + a_2\left(\gamma_{\mu\nu}\Box^{(\mathrm{B})} - \partial_\mu\partial_\nu\right)R^{(2)} - \recip{2}R^{(2)}\gamma_{\mu\nu} + a_3\left(\gamma_{\mu\nu}\Box^{(\mathrm{B})} - \partial_\mu\partial_\nu\right){R^{(1)}}^2 \nonumber \\
 & - {} \recip{6}\overline{h}_{\mu\nu}R^{(1)} - a_2\gamma_{\mu\nu}\overline{h}^{\sigma\rho}\partial_\sigma\partial_\rho R^{(1)} + \frac{a_2}{2} \partial^\rho R^{(1)} \left(\partial_\mu\overline{h}_{\rho\nu} + \partial_\nu\overline{h}_{\rho\mu} - \partial_\rho\overline{h}_{\mu\nu}\right) \nonumber \\
 & + {} a_2\left(R^{(1)}{R^{(1)}}_{\mu\nu} + \recip{4}{R^{(1)}}^2\gamma_{\mu\nu}\right) - {a_2}^2\left(\partial_\mu R^{(1)}\partial_\nu R^{(1)} + \recip{2} \gamma_{\mu\nu}\partial^\rho R^{(1)}\partial_\rho R^{(1)}\right).
\end{align}
It is simplest to split this up for the purposes of averaging. Since we average over all directions at each point gradients average to zero\cite{Hobson2006}
\begin{equation}
\left\langle\partial_\mu V\right\rangle = 0.
\end{equation}
As a corollary of this we have the relation
\begin{equation}
\left\langle U\partial_\mu V\right\rangle = -\left\langle \partial_\mu U V\right\rangle.
\end{equation}
Repeated application of this, together with our gauge condition, \eqnref{Lorenz}, and wave equations, \eqref{eq:Box_R} and \eqref{eq:Box_hmunu}, allows us to eliminate many terms. Considering terms that do not trivially average to zero
\begin{align}
\left\langle {R^{(2)}}_{\mu\nu} \right\rangle = {} & \left\langle -\recip{4} \partial_\mu\overline{h}_{\sigma\rho}\partial^\mu\overline{h}^{\rho\sigma} + \frac{{a_2}^2}{2}\partial_\mu R^{(1)}\partial_\nu R^{(1)} + \frac{a_2}{6}\gamma_{\mu\nu}R^{(1)} \right\rangle; \\
\left\langle R^{(2)} \right\rangle = {} & \left\langle \frac{3a_2}{2}{R^{(1)}}^2 \right\rangle; \\
\left\langle \overline{h}_{\mu\nu}R^{(1)} \right\rangle = {} & 0; \\
\left\langle R^{(1)}{R^{(1)}}_{\mu\nu} \right\rangle = {} & \left\langle a_2 R^{(1)} \partial_\mu\partial_\nu R^{(1)} + \recip{6}\gamma_{\mu\nu}{R^{(1)}}^2\right\rangle.
\end{align}
Combining these gives
\begin{equation}
\left\langle {\mathcal{G}^{(2)}}_{\mu\nu}\right\rangle = \left\langle -\recip{4} \partial_\mu\overline{h}_{\sigma\rho}\partial^\mu\overline{h}^{\rho\sigma} - \frac{3{a_2}^2}{2}\partial_\mu R^{(1)}\partial_\nu R^{(1)} \right\rangle.
\end{equation}
Thus we obtain the result
\begin{equation}
t_{\mu\nu} = \frac{c^4}{32\pi G}\left\langle \partial_\mu\overline{h}_{\sigma\rho}\partial^\mu\overline{h}^{\rho\sigma} + 6{a_2}^2\partial_\mu R^{(1)}\partial_\nu R^{(1)} \right\rangle.
\end{equation}
In the limit of $a_2 \rightarrow 0$ we obtain the standard GR result as required. Note that the GR result is also recovered if $R^{(1)} = 0$, as would be the case if the Ricci mode was not excited, for example if the frequency was below the cut off frequency $\Omega_R$. Rewriting the pseudotensor in terms of metric perturbation $h_{\mu\nu}$, using \eqnref{gauge}, we obtain
\begin{equation}
t_{\mu\nu} = \frac{c^4}{32\pi G}\left\langle \partial_\mu h_{\sigma\rho}\partial^\mu h^{\rho\sigma} + \recip{8}\partial_\mu h \partial_\nu h \right\rangle.
\end{equation}
Note that these results do not depend upon $a_3$ or higher order coefficients.

It might be hoped that these formulae could be used to constrain the parameter $a_2$ through observation of the energy and momentum carried away by gravitational radiation, see \secref{Fifth} for further discussion.

\section{$f(R)$ With A Source}

Having consider radiation in a vacuum, we now move on to consider the case with a source term. We want a first order perturbation from our background metric so that the linearized field equation is
\begin{equation}
{\mathcal{G}^{(1)}}_{\mu\nu} = \frac{8\pi G}{c^4}T_{\mu\nu}.
\end{equation}
We will again assume a Minkowski background, considering terms to first order only. To solve the wave equations \eqref{eq:Box_R} and \eqref{eq:Box_hmunu} with this source term we may use a Green's function
\begin{equation}
\left(\Box + \kappa^2\right)\mathscr{G}_\kappa(x, x') = \delta(x - x'),
\end{equation}
where $\Box$ acts on $x$. The Green's function is familiar as the Klein-Gordon propagator (up to a factor of $-i$)\cite{Peskin1995a}
\begin{equation}
\mathscr{G}_\kappa(x, x') = \int \frac{\dd^4 p}{(2\pi)^4} \frac{\exp\left[-ip\cdot(x-x')\right]}{\kappa^2 - p^2}.
\end{equation}
This may be evaluated by a suitable contour integral to give
\begin{equation}
\mathscr{G}_\kappa(x, x') =
\begin{cases}
{\displaystyle \int{\frac{\dd \omega}{2\pi c} \exp\left[-i\omega(t-t')\right]\recip{4\pi r}\exp\left[i\left(\frac{\omega^2}{c^2} - \kappa^2\right)^{1/2}r\right]}} & \omega^2 > \Omega_R^2\vspace{0.8mm}\\
{\displaystyle \int{\frac{\dd \omega}{2\pi c} \exp\left[-i\omega(t-t')\right]\recip{4\pi r}\exp\left[-\left(\kappa^2 - \frac{\omega^2}{c^2}\right)^{1/2}r\right]}} & \omega^2 < \Omega_R^2\vspace{0.8mm}
\end{cases}\, ,
\label{eq:Green}
\end{equation}
where we have introduced $t = x^0$, $t' = x'^0$ and $r = |\boldsymbol{x} - \boldsymbol{x'}|$. For $\kappa = 0$
\begin{equation}
\mathscr{G}_0(x, x') = \frac{\delta(ct - ct' - r)}{4 \pi c r},
\end{equation}
the standard retarded-time Green's function. We can use this to solve \eqnref{Box_hmunu}
\begin{align}
\overline{h}_{\mu\nu}(x) = {} & -\frac{16 \pi G}{c^4}\int \dd^4 y\, \mathscr{G}_0(x, y) T_{\mu\nu}(x') \nonumber \\
 = {} & -\frac{4 G}{c^4}\int \dd^3 x' \frac{T_{\mu\nu}(ct - r, \boldsymbol{x'})}{r}.
\end{align}
This is exactly as in GR, so we may use standard results.

Solving for scalar mode we find
\begin{equation}
R^{(1)}(x) = -\frac{8 \pi G \kappa^2}{c^4}\int \dd^4 y\, \mathscr{G}_\kappa(x, x') T(x').
\end{equation}
To proceed further we must know the form of the trace $T(y)$. In general the form of $R^{(1)}(x)$ will be complicated.

%For illustrative purposes, let us consider the Newtonian example of two equal mass particles in cicular orbits. Picking our coordinate origin at the centre of the circle, the trace of the stress-energy tensor is
%\begin{equation}
%T(x') = c^2M\left[\delta(x;^1 - a\cos\Omega t')\delta(x'^2 - a\sin\Omega t') + \delta(x'^1 + a\cos\Omega t')\delta(x'^2 + a\sin\Omega t')\right]\delta(x'^3),
%\end{equation}
%where $M$ is the mass, $a$ is the radius and $\Omega$ is the orbital frequency. Using this
%\begin{equation}
%R^{(1)}(x) = -\frac{4 G \kappa^2 M}{c^2}\int \dd t' \int \frac{\dd \omega}{2\pi} \exp\left[-i\omega(t-t')\right]\left\[\frac{\exp\left(i\sqrt{\nicefrac{\Omega^2}{c^2} - \kappa^2}r_+(t')\right)}{r_+(t')} + \right\frac{\exp\left(i\sqrt{\nicefrac{\Omega^2}{c^2} - \kappa^2}r_-(t')\right)}{r_-(t')}},
%\end{equation}
%where we have introduced
%\begin{equation}
%r_\pm^2(t') = (x^1 \pm a\cos\Omega t')^2 + (x^2 \pm a\sin\Omega t')^2.
%\end{equation}
%We may easily evaluate this in the limit of $r \rightarrow \infty$ when we may approximate $r_\pm \simeq r_0 = |\boldsymbol{x}|$. In this limit
%\begin{equation}
%R^{(1)}(x) \simeq -\frac{8 G \kappa^2 M}{c^2}\frac{\exp\(- \kappa r_0)}{r_0}.
%\end{equation}
%We see that 

\subsection{The Newtonian Limit}

Let us consider the limiting case of a Newtonian source such that
\begin{equation}
T_{00} = c^2\rho; \quad |T_{00}| \gg |T_{0i}|; \quad |T_{00}| \gg |T_{ij}|,
\end{equation}
with the simplest mass distribution of a stationary point source
\begin{equation}
\rho = M\delta(\boldsymbol{x'}).
\end{equation}
With this source we do not produce any radiation. As in GR we find
\begin{equation}
\overline{h}_{00} = -\frac{4GM}{c^2r}; \quad \overline{h}_{0i} = \overline{h}_{ij} = 0.
\end{equation}
Solving for the Ricci scalar term gives
\begin{equation}
R^{(1)} = -\frac{2 G \kappa^2 M}{c^2}\frac{\exp(- \kappa r)}{r}.
\end{equation}
Combining these in \eqnref{h_metric} gives a metric perturbation with non-zero elements 
\begin{equation}
h_{00} = -\frac{2GM}{c^2r}\left[1 + \frac{\exp(- \kappa r)}{3}\right]; \quad h_{ii} = -\frac{2GM}{c^2r}\left[1 - \frac{\exp(- \kappa r)}{3}\right] \quad \mathrm{(no\: sum)}.
\end{equation}
Thus, to first order, the metric for a point mass in $f(R)$ gravity is\cite{Capozziello2009a, Naf2010}
\begin{align}
\dd s^2 = & {} \left\{1-\frac{2GM}{c^2r}\left[1 + \frac{\exp(- \kappa r)}{3}\right]\right\}c^2\dd t^2 \nonumber \\
 & - {} \left\{1+\frac{2GM}{c^2r}\left[1 - \frac{\exp(- \kappa r)}{3}\right]\right\}\left(\dd x^2 + \dd y^2 + \dd z^2\right).
\label{eq:f(R)_Schw}
\end{align}
This is not the linearized limit of the Schwarzschild metric, although it is recovered in the limit $a_2 \rightarrow 0$, $\kappa \rightarrow \infty$. Therefore the Schwarzschild solution is not a black hole solution in $f(R)$ gravity\cite{Chiba2007a}. This metric has already been derived for the case of quadratic gravity, which includes terms like $R^2$ and $R_{\mu\nu}R^{\mu\nu}$ in the Lagrangian\cite{Pechlaner1966, Stelle1978, Schmidt1986, Teyssandier1990}. In linearized theory we see that our $f(R)$ is indistinguishable from $f(R) = R + a_2 R$.

We may extend this result for a slowly rotating source with angular momentum $J$, then we have the additional term\cite{Hobson2006}
\begin{equation}
\overline{h}^{0i} = -\frac{2G}{c^2r^3} \epsilon^{ijk}J_j x_k,
\end{equation}
where $\epsilon^{ijk}$ is the alternating Levi-Civita tensor. This gives metric
\begin{align}
\dd s^2 = {} & \left\{1-\frac{2GM}{c^2r}\left[1 + \frac{\exp(- \kappa r)}{3}\right]\right\}c^2\dd t^2 + \frac{4GJ}{c^2r^3}\left(x\dd y - y\dd x\right)\dd t \nonumber \\ & - {} \left\{1 +\frac{2GM}{c^2r}\left[1 - \frac{\exp(- \kappa r)}{3}\right]\right\}\left(\dd x^2 + \dd y^2 + \dd z^2\right),
\end{align}
where we have picked $z$ to be the axis of rotation. This is not the first order limit of the Kerr metric, aside from in the limit $a_2 \rightarrow 0$, $\kappa \rightarrow \infty$.

It has been suggested that since $R = 0$ is a valid solution to the vacuum equations, the black hole solutions of GR should also be solutions in $f(R)$\cite{Psaltis2008, Barausse2008}. However we see here that this is not the case: to have a black hole you must have a source, and, because of \eqnref{Box_R}, this forces $R$ to be non-zero in the surrounding vacuum, although it will decay to zero at infinity\cite{Olmo2007c}. It should therefore be possible to distinguish between theories by observing the black holes that form.

Solving the full field equations to find the exact black hole metric in $f(R)$ is difficult because of the higher order derivatives that enter the equations, but any solution must have the appropriate limiting form as given above.

Additionally, in $f(R)$ gravity Birkhoff's theorem no longer applies: the metric about a spherically symmetric mass does not correspond to the equivalent of the Schwarzschild solution. This is because the distribution of matter influences how the Ricci scalar decays, and consequently Gauss' theorem no longer applies. Repeating our analysis for a (non-rotating) sphere of uniform density and radius $L$ we find that
%\begin{equation}
%\rho(\boldsymbol{x}) = frac{3 M}{4 \pi L^3}\Theta(L - |\boldsymbol{x}|),
%\end{equation}
%where $\Theta$ is the Heaviside step function, 
\begin{equation}
\overline{h}_{00} = -\frac{4GM}{c^2r}; \quad \overline{h}_{0i} = \overline{h}_{ij} = 0.
\end{equation}
as in GR and for the point mass, but
\begin{align}
R^{(1)} = & {} -\frac{6 G M}{c^2}\frac{\exp(- \kappa r)}{r}\left[\frac{\kappa L\cosh(\kappa L) - \sinh(\kappa L)}{\kappa L^3}\right] \\
 = & {} -\frac{6 G M}{c^2}\frac{\exp(- \kappa r)}{r}\kappa^2\Xi(L\kappa),
\end{align}
defining $\Xi(L\kappa)$ in the last line.\footnote{Note that $\Xi(0) = \nicerecip{3}$ is the minimum of $\Xi(L\kappa)$.} The metric perturbation thus has non-zero first order elements\cite{Stelle1978, Capozziello2009b}
\begin{equation}
h_{00} = -\frac{2GM}{c^2r}\left[1 + \exp(- \kappa r)\Xi(L\kappa)\right]; \quad h_{ii} = -\frac{2GM}{c^2r}\left[1 - \exp(- \kappa r)\Xi(L\kappa)\right] \quad \mathrm{(no\: sum)}.
\label{eq:Uniform}
\end{equation}
where we have assumed that $r > L$ at all stages.\footnote{Inside the source $R^{(1)} = -{8 G \kappa^2 M}/{c^2}\left[1 - (\kappa a + 1)\exp(-\kappa a)\sinh(\kappa r)/\kappa r\right]$.}

\subsection{Fifth Force Tests}\label{sec:Fifth}

From the metric \eqnref{f(R)_Schw} we see that a point mass has a Yukawa gravitational potential\cite{Stelle1978, Capozziello2009a}
\begin{equation}
U(r) = \frac{GM}{^2r}\left[1 + \frac{\exp(- \kappa r)}{3}\right].
\end{equation}
Potential of this form are well studied in fifth force tests\cite{Will2006, Adelberger2009, Adelberger2003} which consider a potential defined by coupling constant $\alpha$ and length-scale $\lambdabar$ such that
\begin{equation}
U(r) = \frac{GM}{r}\left[1 + \alpha\exp\left(-\frac{r}{\lambdabar}\right)\right].
\end{equation}
We are able to put strict constraints upon our length-scale $\lambdabar_R$, and hence $a_2$, since our coupling constant $\alpha = \nicerecip{3}$ is relatively large. Note that we would expect this coupling constant to be larger for extended sources: comparison with \eqnref{Uniform} shows that for a uniform sphere $\alpha = \Xi(\kappa L) \geq \nicerecip{3}$.

The best constraints at short distances come from the E\"{o}t-Wash experiments, which use torsion balances\cite{Kapner2007a, Hoyle2004}. These constrain $\lambdabar \lesssim \SI{6e{-5}}{\metre}$. Hence we determine $|a_2| \lesssim \SI{e-9}{\metre}$. A similar result is obtained by N\"{a}f and Jetzer\cite{Naf2010}. This would mean that the cut-off frequency for a propagating scalar mode would be $\Omega_R \gtrsim \SI{5e12}{\Sec^{-1}}$. This is much higher than we would expect for astrophysical objects.

Alternatively $\lambdabar_R$ could be large. Then the exponential would be effectively constant, and we would have to redefine the gravitational constant such that the measured value of Newton's constant is $G\sub{N} \simeq 4G/3$, where $G$ is the bare constant used in the above expressions. Since $\kappa$ would be small, the Ricci scalar would also be small and so we might not notice any deviation from flat space. The best constraints come from planetary precession calculated from Keplerian orbits\cite{Talmadge1988, Adelberger2003}. Extrapolating from these results, for $\alpha \geq \nicerecip{3}$, we must have a minimum length-scale of $\SIrange{e15}{e16}{\metre}$. This would impose the constraint $|a_2| \gtrsim \SI{e30}{\metre^2}$. The cut-off frequency would be $\Omega_R \lesssim \SI{e-7}{\Sec^{-1}}$. The scalar mode could be excited by a variety of astrophysical processes and be detected in the LISA frequency band\cite{Capozziello2008}.

This degeneracy may be broken using other tests. N\"{a}f and Jetzer\cite{Naf2010} consider the precession of orbiting gyroscopes: using the results of Gravity Probe B\cite{Everitt2009} they arrive at the bound $|a_2| \lesssim \SI{e11}{\metre^2}$, and using the double pulsar binary J0737-3039\cite{Burgay2003, Lyne2006, Breton2008} $|a_2| \lesssim \SI{4.6e11}{\metre^2}$. While these are much larger than the E\"{o}t-Wash experiments, they are sufficient to rule out the larger range for $\lambdabar_R$. From this we can conclude that the Ricci scalar mode is unlikely to be excited in astrophysically interesting situations. This unfortunately renders impotent the results of \secref{EM_tensor}.

While the laboratory bound on $\lambdabar_R$ maybe strict compared to astronomical length-scales, it is still much greater than the expected characteristic gravitational scale, the Planck length $\ell\sub{P}$. We might expect for a natural quantum theory, that $a_2 \sim \order{\ell\sub{P}^2}$; however $\ell\sub{P}^2 = \SI{2.612e-70}{\metre^2}$, thus the bound is still about $60$ orders of magnitude greater than the expected value. The only other length-scale that we could introduce would be defined by the cosmological constant $\Lambda$. Using the concordance values\cite{Hinshaw2009} $\Lambda = \SI{1.27e{-52}}{\metre^{-2}}$; we that $\Lambda^{-1} \gg |a_2|$. It is intriguing to note that if we combine these two length-scales we find ${\ell\sub{P}}/{\Lambda^{1/2}} = \SI{1.44e-9}{\metre^2}$, which is on the order of the current bound. This is likely a coincidence, since there is nothing fundamental about the level of current precision, however it would be interesting to see if the measurements could be improved to see if there is any evidence for a Yukawa interaction around this length-scale.

\subsection{The Weak Field Metric}

To continue working with the weak field metric, \eqnref{f(R)_Schw}, it is useful to transform it to the more familiar form
\begin{equation}
\dd s^2 = A(\widetilde{r}) c^2\dd t^2 - B(\widetilde{r})\dd \widetilde{r}^2 + \widetilde{r}^2 \dd \Omega^2.
\label{eq:Sph_sym}
\end{equation}
The coordinate $\widetilde{r}$ is then a circumferential measure as in the Schwarzschild metric as opposed to $r$, used in preceding sections, which is a radial distance, an isotropic coordinate\cite{Misner1973, Olmo2007c}. To simplify the algebra we shall also introduce the Schwarzschild radius
\begin{equation}
r\sub{S} = \frac{2GM}{c^2}.
\end{equation}
In the linearizes regime, we require that the new radial coordinate satisfies
\begin{align}
\widetilde{r}^2 = {} & \left\{1 + \frac{r\sub{S}}{r}\left[1 - \frac{\exp(-\kappa r)}{3}\right]\right\}r^2 \\
\widetilde{r} = {} & r + \frac{r\sub{S}}{2}\left[1 - \frac{\exp(-\kappa r)}{3}\right].
\label{eq:r_tilde}
\end{align}
To first order in ${r\sub{S}}/{r}$\cite{Olmo2007c}
\begin{equation}
A(\widetilde{r}) = 1 - \frac{r\sub{S}}{r}\left[1 + \frac{\exp(-\kappa r )}{3}\right].
\label{eq:A_metric}
\end{equation}
We see that the functional form of $g_{00}$ is almost unchanged upon substituting $\widetilde{r}$ for $r$; however we still have $r$ in the exponential. This must be viewed as being implicitly defined in terms of $\widetilde{r}$.

To find $B(\widetilde{r})$ we consider, using \eqnref{r_tilde},
\begin{align}
\frac{\dd \widetilde{r}}{\widetilde{r}} = {} & \dd \ln \widetilde{r} \nonumber \\
 = {} & \left\{\frac{1 + {\kappa r\sub{S}r\exp(-\kappa r)}/{6\widetilde{r}}}{1 + ({r\sub{S}}/{2\widetilde{r}})\left[1 - {\exp(-\kappa \widetilde{r})}/{3}\right]}\right\}\frac{\dd r}{\widetilde{r}}.
\end{align}
Thus
\begin{equation}
\dd \widetilde{r}^2 = \frac{\widetilde{r}^2}{r^2}\left\{\frac{1 + {\kappa r\sub{S}r\exp(-\kappa r)}/{6\widetilde{r}}}{1 + ({r\sub{S}}/{2\widetilde{r}})\left[1 - {\exp(-\kappa r)}/{3}\right]}\right\}\dd r^2.
\end{equation}
The term in braces is $\left[B(\widetilde{r})\right]^{-1}$. To proceed further we must consider the size of $\kappa r\sub{S}\exp(-\kappa r)$. We are assuming that in the weak field
\begin{equation}
\varepsilon = \frac{r\sub{S}}{r}
\end{equation}
is small. Then the metric perturbations from Minkowski are small. Now
\begin{align}
\kappa r\sub{S}\exp(-\kappa r) = {} & r\varepsilon\kappa\exp(-\kappa r) \nonumber \\
 = {} & \varepsilon\chi\exp(-\chi),
\end{align}
defining $\chi = \kappa r$. The function $\chi\exp(-\chi)$ has a maximum value when $\chi = 1$, hence
\begin{equation}
\kappa r\sub{S}\exp(-\kappa r) \leq \varepsilon\exp(-1).
\end{equation}
So this term is also $\order{\varepsilon}$. We may thus expand to first order\cite{Olmo2007c}
\begin{equation}
B(\widetilde{r})  = 1 + \frac{r\sub{S}}{\widetilde{r}}\left[1 + \frac{\exp(-\kappa r )}{3}\right] - \frac{\kappa r\sub{S} \exp(-\kappa r\sub{S})}{3}.
\label{eq:B_metric}
\end{equation}
In the limit $\kappa \rightarrow \infty$ in which we recover GR, we see that $A(\widetilde{r})$ and $B(\widetilde{r})$ tend to their Schwarzschild forms.

\subsection{Epicyclic Frequencies}\label{sec:Epicycle}

One means of probing the nature of a spacetime is through observations of orbital motions\cite{Gair2008a}. We will consider the epicyclic motion produced by perturbing a circular orbit. We will start by deriving a general result for any metric of the form of \eqnref{Sph_sym}, and then use this for our $f(R)$ solution. For this section we shall adopt units with $c = 1$.

For any metric of the form of \eqnref{Sph_sym} there are three constants of motion: the orbiting particle's rest mass $\mu$, the energy (per unit mass) of the orbit $E$ and the angular momentum (per unit mass) $L$. Denoting differentiation with respect to an affine parameter, which we shall identify as proper time $\tau$ by an over-dot
\begin{align}
E = {} & A\dot{t}; \\
L = {} & \widetilde{r}^2\sin^2\theta\, \dot{\phi}.
\end{align}
As a consequence of the spherical symmetry we may also confine the motion to the equatorial plane $\theta = \pi/2$ without loss of generality. From the Hamiltonian $\mathcal{H} = g_{\mu\nu}\dot{x}^\mu\dot{x}^\nu$ we obtain the equation of motion for massive particles
\begin{equation}
\dot{\widetilde{r}}^2 = \frac{E^2}{AB} - \recip{B}\left(1 + \frac{L^2}{\widetilde{r}^2}\right).
\label{eq:rdot}
\end{equation}
Hence for a circular orbit
\begin{equation}
E^2 = A\left(1 + \frac{L^2}{\widetilde{r}^2}\right).
\end{equation}
Differentiating \eqnref{rdot} yields
\begin{equation}
\ddot{\widetilde{r}} = -\frac{E^2}{2AB}\left(\frac{A'}{A} + \frac{B'}{B}\right) + \frac{B'}{2B^2}\left(1 + \frac{L^2}{\widetilde{r}^2}\right) + \frac{L^2}{\widetilde{r}^3B},
\label{eq:geodesic}
\end{equation}
where a prime signifies differentiation with respect to $\widetilde{r}$. For a circular orbit
\begin{equation}
0 = \frac{2L^2}{\widetilde{r}^3} - \frac{A'}{A}\left(1 + \frac{L^2}{\widetilde{r}^2}\right).
\end{equation}
Thus a circular orbit is defined by one of its energy, angular momentum or radius. We will consider a small perturbation to a circular orbit. Perturbations out of the plane will just redefine the orbital plane, and so are not of interest. A radial perturbation may be parameterized as
\begin{equation}
\widetilde{r} = \overline{r} + \delta,
\end{equation}
where $\overline{r}$ is the radius of our unperturbed orbit. We shall denote $A(\overline{r}) = \overline{A}$ and $B(\overline{r}) = \overline{B}$. Substituting into \eqnref{geodesic} and retaining terms to first order
\begin{equation}
\ddot{\delta} = - \frac{2\overline{A}^2L^2}{\overline{r}^3\overline{A}'\overline{B}}\left(\frac{\overline{A}''}{2\overline{A}^2} - \frac{{\overline{A}'}^2}{\overline{A}^3}\right)\delta + \frac{3L^2}{\overline{r}^4}\delta.
\end{equation}
Assuming a solution of form $\delta = \delta_0\cos(-i\Omega\tau)$,
\begin{equation}
\Omega^2 = \frac{L^2}{\overline{r}^3\overline{B}}\left(\frac{\overline{A}''}{\overline{A}'} - \frac{2\overline{A}'}{\overline{A}} + \frac{3}{\overline{r}}\right).
\end{equation}
We may rewrite the radial motion as
\begin{equation}
\widetilde{r} = \overline{r} + \delta_0\cos(-i\Omega\tau);
\end{equation}
if we compare this with an elliptic Keplerian orbit of small eccentricity $e$
\begin{align}
\widetilde{r} = {} & \frac{a(1 - e^2)}{1 + e\cos(\omega_0\tau)} \\
 = {} & a\left[1 - e\cos(\omega_0\tau) + \ldots \right]
\end{align}
to first order in $e$, where $a$ is the semi-major axis and $\omega_0$ is the orbital frequency; we see we may identify our perturbed orbit with an elliptical orbit where\cite{Kerner2001a}
\begin{equation}
\overline{r} = a; \quad \delta_0 = -ea.
\end{equation}
The eccentricity is the small parameters $|e| = |\delta_0/r| \ll 1$. Notice to this accuracy one cannot distinguish between the $a$ and the seimlatus rectum $p$ as $p = a(1 - e^2)$.

Unless $\omega_0 = \Omega$ the elliptical motion will be asynchronous with the orbital motion: there will be precession of the periapsis. The orbital frequency is
\begin{equation}
\omega_0^2 = \frac{L^2}{\overline{r}^4}.
\end{equation}
In one revolution the ellipse will precess about the focus by
\begin{align}
\upDelta \phi = {} & \omega_0\left(\frac{2\pi}{\Omega} - \frac{2\pi}{\omega_0}\right) \nonumber \\
 = {} & 2\pi\left(\frac{\omega_0}{\Omega} - 1\right)
\end{align}
The precession is cumulative, so a small deviation may be measurable over sufficient time.

For the $f(R)$ metric defined by equations \eqref{eq:A_metric} and \eqref{eq:B_metric} the epicyclic frequency is
\begin{equation}
\Omega^2 = \omega_0^2 \left[1 - \frac{3r\sub{S}}{\overline{r}} - \zeta(\kappa,r\sub{S},\overline{r})\right],
\end{equation}
introducing function
\begin{align}
\zeta(\kappa,r\sub{S},\overline{r}) = {} & r\sub{S}\left(\recip{3\overline{r}} + \kappa\right)\exp(-\kappa r) + \frac{\kappa^2\overline{r}^2\exp(-\kappa r)}{3 + (1 + \kappa \overline{r})\exp(-\kappa r)} \nonumber \\
 & {} \times \left\{1 - \frac{r\sub{S}}{\overline{r}}\left[1 + \frac{\exp(-\kappa r)}{3}\right] - \frac{\kappa r\sub{S}\exp(-\kappa r)}{3}\right\}.
\end{align}
This is the deviation from the Schwarzschild case.

\section{Discussion \& Remaining Questions}


