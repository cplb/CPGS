\chapter{Gravitational Radiation In $f(R)$ Theory}

\section{Introduction To $f(R)$ Theory}

General relativity is a well tested theory of gravity\cite{Will2006}; however it is still interesting to explore alternative theories. This can be motivated by the need to explain dark matter and dark energy in cosmology, trying to formulate a quantizable theory of gravity, or simple curiosity regarding the uniqueness of GR. One of the simplest extensions to standard GR is the class of $f(R)$ theories\cite{Sotiriou2010, DeFelice2010}.

\subsection{The Action \& Field Equations}\label{sec:Action}

General relativity may be derived from the Einstein-Hilbert action\cite{Misner1973, Landau1975}
\begin{equation}
S\sub{EH}[g] = \frac{c^4}{16\pi G}\intd{}{}{R\sqrt{-g}}{^4x}.
\end{equation}
In $f(R)$ theory we make a simple modification of the action to include an arbitrary function of the Ricci scalar $R$ such that\cite{Buchdahl1970}
\begin{equation}
S[g] = \frac{c^4}{16\pi G}\intd{}{}{f(R)\sqrt{-g}}{^4x}.
\end{equation}
Including the function $f(R)$ gives extra freedom in defining the behaviour of gravity; while this action may not encode the true theory of gravity it may at least contain sufficient information to act as an effective field theory, correctly describing phenomenological behaviour\cite{Park2010}. We will assume that $f(R)$ is analytic about $R = 0$ so that it may be expressed as a power series\cite{Buchdahl1970, Psaltis2008}
\begin{equation}
f(R) = a_0 + a_1 R + \frac{a_2}{2!}R^2 + \frac{a_3}{3!}R^3 + \ldots
\end{equation}
Since the dimensions of $f(R)$ must be the same as of $R$, $[a_n] = [R]^{(1-n)}$. To link to GR we will set $a_1 = 1$; any rescaling may be absorbed into the definition of $G$.

The field equations are obtained by a variational principle; there are several ways of achieving this. To derive the Einstein field equations from the Einstein-Hilbert action one may use the standard metric variation or the Palatini variation\cite{Misner1973}. Both approaches may be used for $f(R)$, however they yield different results\cite{Sotiriou2010, DeFelice2010}. Following the metric formalism, one varies the action with respect to the metric $g^{\mu\nu}$, the resulting field equations being for metric $f(R)$-gravity. Following the Palatini formalism one varies the action with respect to both the metric $g^{\mu\nu}$ and the connection ${\Gamma^\rho}_{\mu\nu}$, which are treated as independent quantities: the connection is not the Levi-Civita metric connection.\footnote{Requiring that the metric and Palatini formalisms produce the same field equations, assuming an action that only depends on the metric and Riemann tensor, results in Lovelock gravity\cite{Exirifard2008}. Lovelock gravities require the field equations to be divergence free and no more than second order; in four dimensions the only possible Lovelock gravity is GR with a potentially non-zero cosmological constant\cite{Lovelock1970, Lovelock1971, Lovelock1972}.}

Finally, there is a third version of $f(R)$-gravity: metric-affine $f(R)$-gravity\cite{Sotiriou2007, Sotiriou2007b}. This goes beyond the Palatini formalism by supposing that the matter action is dependent on the variational independent connection. Parallel transport and the covariant derivative are divorced from the metric. This theory has its attractions: it allows for a natural introduction of torsion. However, it is not a metric theory of gravity and so cannot satisfy all the postulates of the Einstein equivalence principle\cite{Will2006}: a free particle does not necessarily follow a geodesic and so the effects of gravity might not be locally removed\cite{Exirifard2008}. The implications of this have not been fully explored, but for this reason we shall not consider the theory further.

We shall restrict our attention to metric $f(R)$-gravity. This is preferred as the Palatini formalism has undesirable properties: static spherically symmetric objects described by a polytropic equation of state are subject to a curvature singularity\cite{Barausse2008b, Barausse2008a}. Varying the action with respect to the metric $g^{\mu\nu}$ produces
\begin{equation}
\delta S = \frac{c^4}{16\pi G}\intd{}{}{\left\{f'(R)\sqrt{-g}\left[R_{\mu\nu} - \nabla_\mu\nabla_\nu\ + g_{\mu\nu}\Box\right] - f(R)\frac{1}{2}\sqrt{-g}g_{\mu\nu}\right\}\delta g^{\mu\nu}}{^4x},
\end{equation}
where $\Box = g^{\mu\nu}\nabla_\mu\nabla_\nu$ is the d'Alembertian and a prime denotes differentiation with respect to $R$. Proceeding from here requires certain assumptions regarding surface terms. In the case of the Einstein-Hilbert action these gather into a total derivative. It is possible to subtract this from the action to obtain a well-defined variational quantity\cite{York1972, Gibbons1977}. This is not the case for general $f(R)$\cite{Madsen1989}. However, since the action includes higher-order derivatives of the metric we are at liberty to fix more degrees of freedom at the boundary, in so doing eliminating the importance of the surface terms\cite{Dyer2009a, Sotiriou2010}. There is no well described prescription for this so we proceed directly to the field equations.

The vacuum field equations are
\begin{equation}
f'R_{\mu\nu} - \nabla_\mu\nabla_\nu f' + g_{\mu\nu}\Box f' - \frac{f}{2}g_{\mu\nu} = 0.
\label{eq:Field_eq}
\end{equation}
Taking the trace of our field equation gives
\begin{equation}
f'R + 3\Box f' - 2f = 0.
\label{eq:Trace_eq}
\end{equation}
If we consider a uniform flat spacetime $R = 0$, this equation gives
\begin{equation}
a_0 = 0.
\label{eq:a_0}
\end{equation}
In analogy to the Einstein tensor, we shall define
\begin{equation}
\mathcal{G}_{\mu\nu} = f'R_{\mu\nu} - \nabla_\mu\nabla_\nu f' + g_{\mu\nu}\Box f' - \frac{f}{2}g_{\mu\nu},
\label{eq:G_tensor}
\end{equation}
so that in a vacuum
\begin{equation}
\mathcal{G}_{\mu\nu} = 0.
\end{equation}

\subsection{Conservation Of Energy-Momentum}

If we introduce matter with a stress-energy tensor $T_{\mu\nu}$, the field equations become
\begin{equation}
\mathcal{G}_{\mu\nu} = \frac{8\pi G}{c^4}T_{\mu\nu}.
\end{equation}
If we act upon this with the covariant derivative we obtain
\begin{align}
\frac{8\pi G}{c^4}\nabla^\mu T_{\mu\nu} = {} & \nabla^\mu\mathcal{G}_{\mu\nu} \nonumber \\
= {} & R_{\mu\nu}\nabla^\mu f' + f'\nabla^\mu\left(R_{\mu\nu} - \recip{2}R g_{\mu\nu}\right) - \left(\Box\nabla_\nu - \nabla_\nu\Box\right)f'.
\end{align}
The second term contains the covariant derivative of the Einstein tensor and so is zero. The final term can be shown to be
\begin{align}
\left(\Box\nabla_\nu - \nabla_\nu\Box\right)f' = {} & g^{\mu\sigma}\left[\nabla_\mu\nabla_\sigma\nabla_\nu - \nabla_\nu\nabla_\mu\nabla_\sigma\right]f' \nonumber \\
 = {} & R_{\tau\nu}\nabla^\tau f',
\end{align}
which is a useful geometric identity\cite{Koivisto2006a}. Using this
\begin{align}
\frac{8\pi G}{c^4}\nabla^\mu T_{\mu\nu} = {} & R_{\mu\nu}\nabla^\mu f' - R_{\mu\nu}\nabla^\mu f' \nonumber \\
 = {} & 0.
\end{align}
Consequently energy-momentum is a conserved quantity in the same way as in GR, as may be expected from the symmetries of the action.

\section{Linearized Theory}\label{sec:Lin}

We will start our investigation of $f(R)$ by looking at linearized theory. This is a weak-field approximation that assumes only small deviations from a flat background, greatly simplifying the field equations. Just as in GR, the linearized framework provides a natural way to study gravitational waves. We will see that the linearized field equations will reduce down to flat-space wave equations: GWs are as much a part of $f(R)$-gravity as of GR.

Consider the case that the metric is perturbed slightly from flat Minkowski space such that
\begin{equation}
g_{\mu\nu} = \eta_{\mu\nu} + h_{\mu\nu};
\end{equation}
where, more formally, we mean that $h_{\mu\nu} = \varepsilon H_{\mu\nu}$ for a small parameter $\varepsilon$.\footnote{It is because we wish to perturb about flat spacetime that we have required $f(R)$ to be analytic about $R = 0$.} We will consider terms only to $\order{\varepsilon}$. Thus, the inverse metric is
\begin{equation}
g^{\mu\nu} = \eta^{\mu\nu} - h^{\mu\nu},
\end{equation}
where we have used the Minkowski metric to raise the indices on the right, defining
\begin{equation}
h^{\mu\nu} = \eta^{\mu\sigma}\eta^{\nu\rho}h_{\sigma\rho}.
\end{equation}
Similarly, the trace $h$ is given by
\begin{equation}
h = \eta^{\mu\nu}h_{\mu\nu}.
\end{equation}
All quantities denoted by ``$h$'' are strictly $\order{\varepsilon}$.

The linearized connection is
\begin{equation}
{{\Gamma^{(1)}}^\rho}_{\mu\nu} = \frac{1}{2}\eta^{\rho\lambda}(\partial_\mu h_{\lambda\nu} + \partial_\nu h_{\lambda\mu} - \partial_\lambda h_{\mu\nu}).
\label{eq:Lin_Gamma}
\end{equation}
To $\order{\varepsilon}$ the covariant derivative of any perturbed quantity will be the same as the partial derivative. The Riemann tensor is
\begin{equation}
{{R^{(1)}}^\lambda}_{\mu\nu\rho} = \frac{1}{2}(\partial_\mu\partial_\nu h^\lambda_\rho + \partial^\lambda\partial_\rho h_{\mu\nu} - \partial_\mu\partial_\rho h^\lambda_\nu - \partial^\lambda\partial_\nu h_{\mu\rho}),
\label{eq:Lin_Riemann}
\end{equation}
where we have raised the index on the differential operator with the background Minkowski metric. Contracting gives the Ricci tensor
\begin{equation}
{R^{(1)}}_{\mu\nu} = \frac{1}{2}(\partial_\mu\partial_\rho h^\rho_\nu + \partial_\nu\partial_\rho h^\rho_\mu -\Box h_{\mu\nu} - \partial_\mu\partial_\nu h),
\label{eq:Ricci}
\end{equation}
where the d'Alembertian operator is $\Box = \eta^{\mu\nu}\partial_\mu\partial_\nu$. Contracting this with $\eta^{\mu\nu}$ gives the first order Ricci scalar
\begin{equation}
R^{(1)} = \partial_\mu\partial_\rho h^{\rho\mu} - \Box h.
\label{eq:Scalar}
\end{equation}

Since $R^{(1)}$ is $\order{\varepsilon}$ we may write $f(R)$ as a Maclaurin series to first order such that
\begin{align}
f(R) = {} & a_0 + R^{(1)};\\
f'(R) = {} & 1 + a_2 R^{(1)}.
\end{align}
As we are perturbing from a Minkowski background where the Ricci scalar vanishes, we may use \eqnref{a_0} to set $a_0 = 0$. Inserting these into \eqnref{G_tensor} and retaining terms to $\order{\varepsilon}$ yields
\begin{equation}
{\mathcal{G}^{(1)}}_{\mu\nu} = {R^{(1)}}_{\mu\nu} - \partial_\mu\partial_\nu(a_2 R^{(1)}) + \eta_{\mu\nu}\Box(a_2 R^{(1)}) - \frac{R^{(1)}}{2}\eta_{\mu\nu}.
\label{eq:Field}
\end{equation}
We need to find a relation between $R^{(1)}$ and its derivatives; let us consider the linearized trace equation, from \eqnref{Trace_eq}
\begin{align}
\mathcal{G}^{(1)} = {} & R^{(1)} + 3 \Box(a_2 R^{(1)}) - 2 R^{(1)} \nonumber \\
\mathcal{G}^{(1)} = {} & 3a_2 \Box R^{(1)} - R^{(1)},
\label{eq:Box_R}
\end{align}
where $\mathcal{G}^{(1)} = \eta^{\mu\nu}{\mathcal{G}^{(1)}}_{\mu\nu}$. This is the massive inhomogeneous Klein-Gordon equation. Setting $\mathcal{G} = 0$, as for a vacuum, we obtain the standard Klein-Gordon equation
\begin{equation}
\Box R^{(1)} + \Upsilon^2 R^{(1)} = 0,
\end{equation}
defining the reciprocal length (squared)
\begin{equation}
\Upsilon^2 = -\recip{3a_2}.
\end{equation}
For a physically meaningful solution $\Upsilon^2 > 0$, thus we constrain $f(R)$ such that $a_2 < 0$\cite{Schmidt1986, Teyssandier1990, Olmo2005c, Corda2007}. From $\Upsilon$ we may define a reduced Compton wavelength
\begin{equation}
\lambdabar_R = \recip{\Upsilon},
\end{equation}
and mass
\begin{equation}
m_R = \frac{\hbar\Upsilon}{c}
\end{equation}
associated with this scalar mode.

The next step is to substitute in $h_{\mu\nu}$ to try to find wave solutions. We want a quantity $\overline{h}_{\mu\nu}$ that will satisfy a wave equation, related to $h_{\mu\nu}$ by
\begin{equation}
\overline{h}_{\mu\nu} = h_{\mu\nu} + A_{\mu\nu}.
\end{equation}
In GR we use the trace-reversed form where $A_{\mu\nu} = -(h/2)\eta_{\mu\nu}$. This will not suffice here, but let us look for a similar solution
\begin{equation}
\overline{h}_{\mu\nu} = h_{\mu\nu} - \frac{h}{2}\eta_{\mu\nu} + B_{\mu\nu}.
\end{equation}
The only rank two tensors in our theory are: $h_{\mu\nu}$, $\eta_{\mu\nu}$, ${R^{(1)}}_{\mu\nu}$, and $\partial_\mu\partial_\nu$; $h_{\mu\nu}$ has been used already, and we wish to eliminate ${R^{(1)}}_{\mu\nu}$, so we will try the simpler option based around $\eta_{\mu\nu}$. We want $B_{\mu\nu}$ to be $\order{\varepsilon}$. There are three scalar quantities that satisfy this: $h$, $R^{(1)}$ and $\Box R^{(1)}$; $h$ is used already and $\Box R^{(1)}$ is related to $R^{(1)}$ by \eqnref{Box_R}. Therefore, we may construct an ansatz
\begin{equation}
\overline{h}_{\mu\nu} = h_{\mu\nu} + \left(b a_2 R^{(1)} - \frac{h}{2}\right)\eta_{\mu\nu},
\label{eq:Ansatz}
\end{equation}
where $a_2$ has been included to ensure dimensional consistency and $b$ is a dimensionless number. Contracting with the background metric yields
\begin{equation}
\overline{h} = 4b a_2 R^{(1)} - h,
\label{eq:h_trace}
\end{equation}
so we may eliminate $h$ in our definition of $\overline{h}_{\mu\nu}$ to give
\begin{equation}
h_{\mu\nu} = \overline{h}_{\mu\nu} + \left(b a_2 R^{(1)} -\frac{\overline{h}}{2}\right)\eta_{\mu\nu}.
\end{equation}
Just as in GR, we have the freedom to perform a gauge transformation\cite{Misner1973, Hobson2006}: the field equations are gauge invariant since we started with a function of the gauge invariant Ricci scalar. We will assume a Lorenz, or de Donder, gauge choice so
\begin{equation}
\nabla^\mu \overline{h}_{\mu\nu} = 0;
\label{eq:Lorenz}
\end{equation}
to first order this is
\begin{equation}
\partial^\mu \overline{h}_{\mu\nu} = 0.
\end{equation}
Subject to this, from \eqnref{Ricci}, the Ricci tensor is
\begin{equation}
{R^{(1)}}_{\mu\nu} = -\frac{1}{2}\left\{2b a_2 \partial_\mu\partial_\nu R^{(1)} + \Box\left(\overline{h}_{\mu\nu} -\frac{\overline{h}}{2}\eta_{\mu\nu}\right) + \frac{b}{3}(R^{(1)} + \mathcal{G}^{(1)})\eta_{\mu\nu}\right\}.
\end{equation}
Using this with \eqnref{Box_R} in \eqnref{Field} gives
\begin{equation}
-\frac{1}{2}\Box\left(\overline{h}_{\mu\nu} - \frac{\overline{h}}{2}\eta_{\mu\nu}\right) - (b + 1)\left(a_2\partial_\mu\partial_\nu R^{(1)} + \frac{R^{(1)}}{6}\eta_{\mu\nu}\right) = {\mathcal{G}^{(1)}}_{\mu\nu} + \frac{b - 2}{6}\mathcal{G}^{(1)}\eta_{\mu\nu}.
\label{eq:b_Field}
\end{equation}
Picking $b = -1$ the second term vanishes, thus we will set\cite{Corda2007, Capozziello2008}
\begin{align}
\overline{h}_{\mu\nu} = {} & h_{\mu\nu} - \left(a_2 R^{(1)} + \frac{h}{2}\right)\eta_{\mu\nu}\\
h_{\mu\nu} = {} & \overline{h}_{\mu\nu} - \left(a_2 R^{(1)} + \frac{\overline{h}}{2}\right)\eta_{\mu\nu}.
\label{eq:h_metric}
\end{align}
From \eqnref{Scalar} the Ricci scalar in this case is 
\begin{align}
R^{(1)} = {} & \Box \left(a_2 R^{(1)} -\frac{\overline{h}}{2}\right) - \Box (-4 a_2 R^{(1)} - \overline{h}) \nonumber \\
 = {} & 3a_2 \Box R^{(1)} + \frac{1}{2}\Box \overline{h}.
\label{eq:Ricci_Box_h}
\end{align}
For consistency with \eqnref{Box_R}, we require
\begin{equation}
-\recip{2}\Box \overline{h} = \mathcal{G}^{(1)}.
\label{eq:Box_h}
\end{equation}
Inserting this into \eqnref{b_Field}, with $b = -1$, we see
\begin{equation}
-\recip{2}\Box \overline{h}_{\mu\nu} = {\mathcal{G}^{(1)}}_{\mu\nu};
\label{eq:Box_hmunu}
\end{equation}
we have our wave equation.

Should $a_2$ be sufficiently small that it may be regarded an $\order{\varepsilon}$ quantity, we recover GR to leading order within our analysis.

\section{Gravitational Radiation}

Having established two wave equations, \eqref{eq:Box_R} and \eqref{eq:Box_hmunu}, we may now investigate their solutions. We shall consider waves in a vacuum such that $\mathcal{G}_{\mu\nu} = 0$. Using a standard Fourier decomposition
\begin{align}
\overline{h}_{\mu\nu} = {} & \widehat{\overline{h}}_{\mu\nu}(k_\rho) \exp\left(ik_\rho x^\rho\right),\\
R^{(1)} = {} & \widehat{R}(q_\rho) \exp\left(iq_\rho x^\rho\right),
\end{align}
where $k_\mu$ and $q_\mu$ are 4-wavevectors. From \eqnref{Box_hmunu} we know that $k_\mu$ is a null vector, so for a wave travelling along the $z$-axis
\begin{equation}
k^\mu = \frac{\omega}{c}(1, 0, 0, 1),
\end{equation}
where $\omega$ is the angular frequency. Similarly, from \eqnref{Box_R}
\begin{equation}
q^\mu = \left(\frac{\Omega}{c}, 0, 0, \sqrt{\frac{\Omega^2}{c^2} - \Upsilon^2}\right),
\label{eq:Ricci_q}
\end{equation}
for frequency $\Omega$. These waves do not travel at $c$, but have a group velocity
\begin{equation}
v = \frac{c\sqrt{\Omega^2 - c^2\Upsilon^2}}{\Omega},
\end{equation}
provided that $\Upsilon^2 > 0$, $v < c$. For $\Omega < \Omega_R = c\Upsilon$, we will find an evanescently decaying wave instead of a propagating mode.

From the condition \eqref{eq:Lorenz} we find that $k^\mu$ is orthogonal to $\widehat{\overline{h}}_{\mu\nu}$,
\begin{equation}
k^\mu\widehat{\overline{h}}_{\mu\nu} = 0,
\end{equation}
thus in this case
\begin{equation}
\widehat{\overline{h}}_{0\nu} + \widehat{\overline{h}}_{3\nu} = 0.
\label{eq:Transverse}
\end{equation}

Let us now consider the implications of \eqnref{Box_h} using equations \eqref{eq:h_trace} and \eqref{eq:Box_R},
\begin{align}
\Box\left(4a_2R^{(1)} + h\right) = {} & 0 \nonumber \\
\Box h = {} & -\frac{4}{3}R^{(1)}.
\end{align}
For non-zero $R^{(1)}$ (as required for the Ricci mode) there is no way we can make a gauge choice such that the trace $h$ will vanish\cite{Corda2007, Capozziello2008}. This is distinct from in GR. It is possible, however, to make a gauge choice such that the trace $\overline{h}$ will vanish. Consider a gauge transformation generated by $\xi_\mu$ which satisfies $\Box \xi_\mu = 0$, and so has a Fourier decomposition
\begin{equation}
\xi_\mu = \widehat{\xi}_\mu \exp\left(ik_\rho x^\rho\right).
\end{equation}
A transformation
\begin{equation}
\overline{h}_{\mu\nu} \rightarrow \overline{h}_{\mu\nu} + \partial_\mu\xi_\nu + \partial_\nu\xi_\mu - \eta_{\mu\nu}\partial^\rho\xi_\rho,
\end{equation}
would ensure both conditions \eqref{eq:Lorenz} and \eqref{eq:Box_hmunu} are satisfied\cite{Misner1973}. Under such a transformation
\begin{equation}
\widehat{\overline{h}}_{\mu\nu} \rightarrow \widehat{\overline{h}}_{\mu\nu} + i\left(k_\mu\widehat{\xi}_\nu + k_\nu\widehat{\xi}_\mu - \eta_{\mu\nu}k^\rho\widehat{\xi}_\rho\right).
\end{equation}
We may therefore impose four further constraints (one for each $\widehat{\xi}_\mu$) upon $\widehat{\overline{h}}_{\mu\nu}$. We take these to be
\begin{equation}
\widehat{\overline{h}}_{0\nu} = 0, \quad \widehat{\overline{h}} = 0.
\end{equation}
This may appear to be five constraints, however we have already imposed \eqref{eq:Transverse}, and so setting $\widehat{\overline{h}}_{00} = 0$ automatically implies $\widehat{\overline{h}}_{03} = 0$. In this gauge we have
\begin{align}
h_{\mu\nu} = {} & \overline{h}_{\mu\nu} - a_2 R^{(1)}\eta_{\mu\nu},\\
h = {} & -4a_2R^{(1)}.
\label{eq:gauge}
\end{align}
Thus $\overline{h}_{\mu\nu}$ behaves just as its GR counterpart so we may define
\begin{equation}
\left[\widehat{\overline{h}}_{\mu\nu}\right] =
\begin{bmatrix}
0 & 0 & 0 & 0\\
0 & h_+ & h_\times & 0\\
0 & h_\times & -h_+ & 0\\
0 & 0 & 0 & 0
\end{bmatrix},
\end{equation}
where $h_+$ and $h_\times$ are constants representing the amplitudes of the two transverse polarizations of gravitational radiation.

It is important that our solutions reduce to those of GR in the event that $f(R) = R$. In this linearized approach this corresponds to $a_2 \rightarrow 0$, $\Upsilon^2 \rightarrow \infty$. We see from \eqnref{Ricci_q} that in this limit it would take an infinite frequency to excite a propagating Ricci mode, and evanescent waves would decay away infinitely quickly. Therefore there would be no detectable Ricci modes and we would only observe the two polarizations found in GR. Additionally $\overline{h}_{\mu\nu}$ would simplify to its usual trace-reversed form.

\section{$f(R)$ With A Source}

Having considered radiation in a vacuum, we now move on to the case with a source term. We want a first order perturbation from our background metric so the linearized field equation is
\begin{equation}
{\mathcal{G}^{(1)}}_{\mu\nu} = \frac{8\pi G}{c^4}T_{\mu\nu}.
\end{equation}
We will again assume a Minkowski background, considering terms to $\order{\varepsilon}$ only. To solve the wave equations \eqref{eq:Box_R} and \eqref{eq:Box_hmunu} with this source term we use a Green's function
\begin{equation}
\left(\Box + \Upsilon^2\right)\mathscr{G}_\Upsilon(x, x') = \delta(x - x'),
\end{equation}
where $\Box$ acts on $x$. The Green's function is familiar as the Klein-Gordon propagator (up to a factor of $-i$)\cite{Peskin1995a}
\begin{equation}
\mathscr{G}_\Upsilon(x, x') = \int \frac{\dd^4 p}{(2\pi)^4} \frac{\exp\left[-ip\cdot(x-x')\right]}{\Upsilon^2 - p^2}.
\end{equation}
This may be evaluated by a suitable contour integral to give
\begin{equation}
\mathscr{G}_\Upsilon(x, x') =
\begin{cases}
{\displaystyle \int{\frac{\dd \omega}{2\pi c} \exp\left[-i\omega(t-t')\right]\recip{4\pi r}\exp\left[i\left(\frac{\omega^2}{c^2} - \Upsilon^2\right)^{1/2}r\right]}} & \omega^2 > \Omega_R^2\vspace{0.8mm}\\
{\displaystyle \int{\frac{\dd \omega}{2\pi c} \exp\left[-i\omega(t-t')\right]\recip{4\pi r}\exp\left[-\left(\Upsilon^2 - \frac{\omega^2}{c^2}\right)^{1/2}r\right]}} & \omega^2 < \Omega_R^2\vspace{0.8mm}
\end{cases}\, ,
\label{eq:Green}
\end{equation}
where we have introduced $ct = x^0$, $ct' = x'^0$ and $r = |\boldsymbol{x} - \boldsymbol{x'}|$. For $\Upsilon = 0$
\begin{equation}
\mathscr{G}_0(x, x') = \frac{\delta(ct - ct' - r)}{4 \pi c r},
\end{equation}
the standard retarded-time Green's function. We can use this to solve \eqnref{Box_hmunu}
\begin{align}
\overline{h}_{\mu\nu}(x) = {} & -\frac{16 \pi G}{c^4}\int \dd^4 x'\, \mathscr{G}_0(x, x') T_{\mu\nu}(x') \nonumber \\
 = {} & -\frac{4 G}{c^4}\int \dd^3 x' \frac{T_{\mu\nu}(ct - r, \boldsymbol{x'})}{r}.
\end{align}
This is exactly as in GR, so we may use standard results.

Solving for the scalar mode
\begin{equation}
R^{(1)}(x) = -\frac{8 \pi G \Upsilon^2}{c^4}\int \dd^4 x'\, \mathscr{G}_\Upsilon(x, x') T(x').
\end{equation}
To proceed further we must know the form of the trace $T(x')$. In general the form of $R^{(1)}(x)$ will be complicated.

%For illustrative purposes, let us consider the Newtonian example of two equal mass particles in circular orbits. Picking our coordinate origin at the centre of the circle, the trace of the stress-energy tensor is
%\begin{equation}
%T(x') = c^2M\left[\delta(x;^1 - a\cos\Omega t')\delta(x'^2 - a\sin\Omega t') + \delta(x'^1 + a\cos\Omega t')\delta(x'^2 + a\sin\Omega t')\right]\delta(x'^3),
%\end{equation}
%where $M$ is the mass, $a$ is the radius and $\Omega$ is the orbital frequency. Using this
%\begin{equation}
%R^{(1)}(x) = -\frac{4 G \Upsilon^2 M}{c^2}\int \dd t' \int \frac{\dd \omega}{2\pi} \exp\left[-i\omega(t-t')\right]\left\[\frac{\exp\left(i\sqrt{\nicefrac{\Omega^2}{c^2} - \Upsilon^2}r_+(t')\right)}{r_+(t')} + \right\frac{\exp\left(i\sqrt{\nicefrac{\Omega^2}{c^2} - \Upsilon^2}r_-(t')\right)}{r_-(t')}},
%\end{equation}
%where we have introduced
%\begin{equation}
%r_\pm^2(t') = (x^1 \pm a\cos\Omega t')^2 + (x^2 \pm a\sin\Omega t')^2.
%\end{equation}
%We may easily evaluate this in the limit of $r \rightarrow \infty$ when we may approximate $r_\pm \simeq r_0 = |\boldsymbol{x}|$. In this limit
%\begin{equation}
%R^{(1)}(x) \simeq -\frac{8 G \Upsilon^2 M}{c^2}\frac{\exp\(- \Upsilon r_0)}{r_0}.
%\end{equation}
%We see that 

\subsection{The Newtonian Limit}

Let us consider the limiting case of a Newtonian source, such that
\begin{equation}
T_{00} = c^2\rho; \quad |T_{00}| \gg |T_{0i}|; \quad |T_{00}| \gg |T_{ij}|,
\end{equation}
with a mass distribution of a stationary point source
\begin{equation}
\rho = M\delta(\boldsymbol{x'}).
\end{equation}
This source does not produce any radiation. As in GR we find
\begin{equation}
\overline{h}_{00} = -\frac{4GM}{c^2r}; \quad \overline{h}_{0i} = \overline{h}_{ij} = 0.
\end{equation}
Solving for the Ricci scalar term gives
\begin{equation}
R^{(1)} = -\frac{2 G \Upsilon^2 M}{c^2}\frac{\exp(- \Upsilon r)}{r}.
\end{equation}
Combining these in \eqnref{h_metric} yields a metric perturbation with non-zero elements 
\begin{equation}
h_{00} = -\frac{2GM}{c^2r}\left[1 + \frac{\exp(- \Upsilon r)}{3}\right]; \quad h_{ii} = -\frac{2GM}{c^2r}\left[1 - \frac{\exp(- \Upsilon r)}{3}\right] \quad \mathrm{(no\: sum)}.
\end{equation}
Thus, to first order, the metric for a point mass in $f(R)$-gravity is\cite{Capozziello2009a, Naf2010}
\begin{align}
\dd s^2 = & {} \left\{1-\frac{2GM}{c^2r}\left[1 + \frac{\exp(- \Upsilon r)}{3}\right]\right\}c^2\dd t^2 \nonumber \\
 & - {} \left\{1+\frac{2GM}{c^2r}\left[1 - \frac{\exp(- \Upsilon r)}{3}\right]\right\}\left(\dd x^2 + \dd y^2 + \dd z^2\right).
\label{eq:f(R)_Schw}
\end{align}
This is not the linearized limit of the Schwarzschild metric, although it is recovered as $a_2 \rightarrow 0$, $\Upsilon \rightarrow \infty$. Therefore the Schwarzschild solution is not a black hole (BH) solution in $f(R)$-gravity\cite{Chiba2007a}. This metric has already been derived for the case of quadratic gravity, which includes terms like $R^2$ and $R_{\mu\nu}R^{\mu\nu}$ in the Lagrangian\cite{Pechlaner1966, Stelle1978, Schmidt1986, Teyssandier1990}. In linearized theory our $f(R)$ reduces to quadratic theory, as to first order $f(R) = R + a_2 R^2$.

We may extend this result to a slowly rotating source with angular momentum $J$; then we have the additional term\cite{Hobson2006}
\begin{equation}
\overline{h}^{0i} = -\frac{2G}{c^2r^3} \epsilon^{ijk}J_j x_k,
\end{equation}
where $\epsilon^{ijk}$ is the alternating Levi-Civita tensor. The metric is
\begin{align}
\dd s^2 = {} & \left\{1-\frac{2GM}{c^2r}\left[1 + \frac{\exp(- \Upsilon r)}{3}\right]\right\}c^2\dd t^2 + \frac{4GJ}{c^2r^3}\left(x\dd y - y\dd x\right)\dd t \nonumber \\ & - {} \left\{1 +\frac{2GM}{c^2r}\left[1 - \frac{\exp(- \Upsilon r)}{3}\right]\right\}\left(\dd x^2 + \dd y^2 + \dd z^2\right),
\end{align}
where $z$ is the rotation axis. This is not the first order limit of the Kerr metric, aside from in the limit $a_2 \rightarrow 0$, $\Upsilon \rightarrow \infty$.

It has been suggested that since $R = 0$ is a valid solution to the vacuum equations, the BH solutions of GR should also be solutions in $f(R)$\cite{Psaltis2008, Barausse2008}. However we see here that this is not the case: to have a BH you must have a source, and, because of \eqnref{Box_R}, this forces $R$ to be non-zero in the surrounding vacuum, although it will decay to zero at infinity\cite{Olmo2007c}. It should therefore be possible to distinguish between theories by observing the BHs that form.

Solving the full field equations to find the exact BH metric in $f(R)$ is difficult because of the higher-order derivatives that enter the equations. Any solution must have the appropriate limiting form as given above.

In $f(R)$-gravity Birkhoff's theorem no longer applies: the metric about a spherically symmetric mass does not correspond to the equivalent of the Schwarzschild solution, since the distribution of matter influences how the Ricci scalar decays, and consequently Gauss' theorem no longer applies. Repeating our analysis for a (non-rotating) sphere of uniform density and radius $L$ we find
%\begin{equation}
%\rho(\boldsymbol{x}) = frac{3 M}{4 \pi L^3}\Theta(L - |\boldsymbol{x}|),
%\end{equation}
%where $\Theta$ is the Heaviside step function, 
\begin{equation}
\overline{h}_{00} = -\frac{4GM}{c^2r}; \quad \overline{h}_{0i} = \overline{h}_{ij} = 0,
\end{equation}
as in GR, and for the point mass, but
\begin{align}
R^{(1)} = & {} -\frac{6 G M}{c^2}\frac{\exp(- \Upsilon r)}{r}\left[\frac{\Upsilon L\cosh(\Upsilon L) - \sinh(\Upsilon L)}{\Upsilon L^3}\right] \\
 = & {} -\frac{6 G M}{c^2}\frac{\exp(- \Upsilon r)}{r}\Upsilon^2\Xi(\Upsilon L),
\end{align}
defining $\Xi(\Upsilon L)$ in the last line.\footnote{$\Xi(0) = \nicerecip{3}$ is the minimum of $\Xi(\Upsilon L)$.} The metric perturbation thus has non-zero first order elements\cite{Stelle1978, Capozziello2009b}
\begin{equation}
h_{00} = -\frac{2GM}{c^2r}\left[1 + \exp(- \Upsilon r)\Xi(\Upsilon L)\right]; \quad h_{ii} = -\frac{2GM}{c^2r}\left[1 - \exp(- \Upsilon r)\Xi(\Upsilon L)\right] \quad \mathrm{(no\: sum)}.
\label{eq:Uniform}
\end{equation}
where we have assumed that $r > L$ at all stages.\footnote{Inside the source $R^{(1)} = -{6 G M}\left[1 - (\Upsilon L + 1)\exp(-\Upsilon L)\sinh(\Upsilon r)/\Upsilon r\right]/{c^2L^3}$.}

\subsection{Fifth-Force Tests}\label{sec:Fifth}

From the metric \eqnref{f(R)_Schw} we see that a point mass has a Yukawa gravitational potential\cite{Stelle1978, Capozziello2009a}
\begin{equation}
U(r) = \frac{GM}{r}\left[1 + \frac{\exp(- \Upsilon r)}{3}\right].
\end{equation}
Potentials of this form are well studied in fifth-force tests\cite{Will2006, Adelberger2009, Adelberger2003} which consider a potential defined by a coupling constant $\alpha$ and a length-scale $\lambdabar$ such that
\begin{equation}
U(r) = \frac{GM}{r}\left[1 + \alpha\exp\left(-\frac{r}{\lambdabar}\right)\right].
\end{equation}
We are able to put strict constraints upon our length-scale $\lambdabar_R$, and hence $a_2$, since our coupling constant $\alpha_R = \nicerecip{3}$ is relatively large. We would expect this coupling constant to be larger for extended sources: comparison with \eqnref{Uniform} shows that for a uniform sphere $\alpha_R = \Xi(\Upsilon L) \geq \nicerecip{3}$.

The best constraints at short distances come from the E\"{o}t-Wash experiments, which use torsion balances\cite{Kapner2007a, Hoyle2004}. These constrain $\lambdabar_R \lesssim \SI{8e-5}{\metre}$. Hence we determine $|a_2| \lesssim \SI{2e-9}{\metre^2}$. A similar result is obtained by N\"{a}f and Jetzer\cite{Naf2010}. This would mean that the cut-off frequency for a propagating scalar mode would be $\Omega_R \gtrsim \SI{4e12}{\second^{-1}}$. This is much higher than expected for astrophysical objects.

Fifth-force tests also permit $\lambdabar_R$ to be large. This degeneracy can be broken using other tests. From \eqnref{f(R)_Schw}, calculating the post-Newtonian parameter $\gamma$, which measures the space-curvature produced by unit rest mass\cite{Will2006}, we find\cite{Olmo2007c, DeFelice2010}
\begin{equation}
\gamma = \frac{3 - \exp(-\Upsilon r)}{3 + \exp(-\Upsilon r)}.
\end{equation}
As $\Upsilon \rightarrow \infty$, the GR value of $\gamma = 1$ is recovered. To be consistent with the current observational values of $\gamma = 1 + (2.1 \pm 2.3) \times 10^{-5}$\cite{Will2006, Bertotti2003} we must require $\Upsilon r \gg 1$ on solar system scales. This excludes the larger range for $\lambdabar_R$. Note that this bound for $\gamma$ was derived assuming that it was independent of position. We will see that the large range for $\lambdabar_R$ is also excluded by planetary perihelion precession in \secref{Epicycle}.

While the laboratory bound on $\lambdabar_R$ may be strict compared to astronomical length-scales, it is still much greater than the expected characteristic gravitational scale, the Planck length $\ell\sub{P}$. We might expect for a natural quantum theory, that $a_2 \sim \order{\ell\sub{P}^2}$; however $\ell\sub{P}^2 = \SI{2.612e-70}{\metre^2}$, thus the bound is still about $60$ orders of magnitude greater than the natural value. The only other length-scale that we could introduce would be defined by the cosmological constant $\Lambda$. Using the concordance values\cite{Hinshaw2009} $\Lambda = \SI{1.27e-52}{\metre^{-2}}$; we see that $\Lambda^{-1} \gg |a_2|$. It is intriguing to note that if we combine these two length-scales we find ${\ell\sub{P}}/{\Lambda^{1/2}} = \SI{1.44e-9}{\metre^2}$, which is on the order of the current bound. This is likely to be a coincidence, since there is nothing fundamental about the level of current precision. It would be interesting to see if the measurements could be improved to rule out a Yukawa interaction around this length-scale.

Here we have only discussed tests in the solar system. It may be, if $f(R)$ is just an effective theory, that the value of $a_2$ is different in different regions. This may allow the Ricci mode to be excited and propagate outside of the solar system. We discuss this more in \secref{f_Discuss}.

\subsection{The Weak-Field Metric}

To continue working with the weak-field metric, \eqnref{f(R)_Schw}, it is useful to transform it to the more familiar form
\begin{equation}
\dd s^2 = A(\widetilde{r}) c^2\dd t^2 - B(\widetilde{r})\dd \widetilde{r}^{\,2} - \widetilde{r}^{\,2} \dd \Omega^2.
\label{eq:Sph_sym}
\end{equation}
The coordinate $\widetilde{r}$ is a circumferential measure, as in the Schwarzschild metric, as opposed to $r$, used in preceding sections, which is a radial distance, an isotropic coordinate\cite{Misner1973, Olmo2007c}. To simplify the algebra we shall introduce the Schwarzschild radius
\begin{equation}
r\sub{S} = \frac{2GM}{c^2}.
\end{equation}
In the linearized regime, we require that the new radial coordinate satisfies
\begin{align}
\widetilde{r}^{\,2} = {} & \left\{1 + \frac{r\sub{S}}{r}\left[1 - \frac{\exp(-\Upsilon r)}{3}\right]\right\}r^2 \\
\widetilde{r} = {} & r + \frac{r\sub{S}}{2}\left[1 - \frac{\exp(-\Upsilon r)}{3}\right].
\label{eq:r_tilde}
\end{align}
To first order in ${r\sub{S}}/{r}$\cite{Olmo2007c}
\begin{equation}
A(\widetilde{r}) = 1 - \frac{r\sub{S}}{\widetilde{r}}\left[1 + \frac{\exp(-\Upsilon r )}{3}\right].
\label{eq:A_metric}
\end{equation}
We see that the functional form of $g_{00}$ is almost unchanged upon substituting $\widetilde{r}$ for $r$; however $r$ is still in the exponential.

To find $B(\widetilde{r})$ we consider, using \eqnref{r_tilde},
\begin{align}
\frac{\dd \widetilde{r}}{\widetilde{r}} = {} & \dd \ln \widetilde{r} \nonumber \\
 = {} & \left\{\frac{1 + {\Upsilon r\sub{S}r\exp(-\Upsilon r)}/{6\widetilde{r}}}{1 + ({r\sub{S}}/{2\widetilde{r}})\left[1 - {\exp(-\Upsilon \widetilde{r})}/{3}\right]}\right\}\frac{\dd r}{\widetilde{r}}.
\end{align}
Thus
\begin{equation}
\dd \widetilde{r}^{\,2} = \frac{\widetilde{r}^{\,2}}{r^2}\left\{\frac{1 + {\Upsilon r\sub{S}r\exp(-\Upsilon r)}/{6\widetilde{r}}}{1 + ({r\sub{S}}/{2\widetilde{r}})\left[1 - {\exp(-\Upsilon r)}/{3}\right]}\right\}\dd r^2.
\end{equation}
The term in braces is $\left[B(\widetilde{r})\right]^{-1}$. To proceed further we must check the size of $\Upsilon r\sub{S}\exp(-\Upsilon r)$. We assume that in the weak-field
\begin{equation}
\varepsilon = \frac{r\sub{S}}{r}
\end{equation}
is small. Then the metric perturbations from Minkowski are small. Now
\begin{align}
\Upsilon r\sub{S}\exp(-\Upsilon r) = {} & r\varepsilon\Upsilon\exp(-\Upsilon r) \nonumber \\
 = {} & \varepsilon\chi\exp(-\chi),
\end{align}
defining $\chi = \Upsilon r$. The function $\chi\exp(-\chi)$ has a maximum value when $\chi = 1$, hence
\begin{equation}
\Upsilon r\sub{S}\exp(-\Upsilon r) \leq \varepsilon\exp(-1).
\end{equation}
This term is also $\order{\varepsilon}$. Expanding to first order\cite{Olmo2007c}
\begin{equation}
B(\widetilde{r})  = 1 + \frac{r\sub{S}}{\widetilde{r}}\left[1 + \frac{\exp(-\Upsilon r )}{3}\right] - \frac{\Upsilon r\sub{S} \exp(-\Upsilon r\sub{S})}{3}.
\label{eq:B_metric}
\end{equation}
In the limit $\Upsilon \rightarrow \infty$, where we recover GR, $A(\widetilde{r})$ and $B(\widetilde{r})$ tend to their Schwarzschild forms.

\subsection{Epicyclic Frequencies}\label{sec:Epicycle}

One means of probing the nature of a spacetime is through observations of orbital motions\cite{Gair2008a}. We will consider the epicyclic motion produced by perturbing a circular orbit. We will start by deriving a general result for any metric of the form of \eqnref{Sph_sym}, and then use this for our $f(R)$ solution. For this section we shall adopt units with $c = 1$.

For any metric of the form of \eqnref{Sph_sym} there are three constants of motion: the orbiting particle's rest mass $\mu$, the energy (per unit mass) of the orbit $E$, and the $z$-component of the angular momentum (per unit mass) $L$. Using an over-dot to denote differentiation with respect to an affine parameter, which we shall identify as proper time $\tau$,
\begin{align}
E = {} & A\dot{t}; \\
L = {} & \widetilde{r}^{\,2}\sin^2\theta\, \dot{\phi}.
\end{align}
As a consequence of the spherical symmetry we may confine the motion to the equatorial plane $\theta = \pi/2$ without loss of generality. From the Hamiltonian $\mathcal{H} = g_{\mu\nu}\dot{x}^\mu\dot{x}^\nu$ we obtain the equation of motion for massive particles
\begin{equation}
\dot{\widetilde{r}}^{\,2} = \frac{E^2}{AB} - \recip{B}\left(1 + \frac{L^2}{\widetilde{r}^{\,2}}\right).
\label{eq:rdot}
\end{equation}
Hence for a circular orbit
\begin{equation}
E^2 = A\left(1 + \frac{L^2}{\widetilde{r}^{\,2}}\right).
\end{equation}
Differentiating \eqnref{rdot} yields
\begin{equation}
\ddot{\widetilde{r}} = -\frac{E^2}{2AB}\left(\frac{A'}{A} + \frac{B'}{B}\right) + \frac{B'}{2B^2}\left(1 + \frac{L^2}{\widetilde{r}^{\,2}}\right) + \frac{L^2}{\widetilde{r}^{\,3}B},
\label{eq:geodesic}
\end{equation}
where a prime signifies differentiation with respect to $\widetilde{r}$. For a circular orbit
\begin{equation}
0 = \frac{2L^2}{\widetilde{r}^{\,3}} - \frac{A'}{A}\left(1 + \frac{L^2}{\widetilde{r}^{\,2}}\right).
\end{equation}
Thus a circular orbit is defined by one of $\{E,L,\widetilde{r}\}$. We will consider a small perturbation to a circular orbit. Perturbations out of the plane just redefine the orbital plane; they are not of interest. A radial perturbation may be parameterized as
\begin{equation}
\widetilde{r} = \overline{r} + \delta,
\end{equation}
where $\overline{r}$ is the radius of the unperturbed orbit. We will denote $A(\overline{r}) = \overline{A}$ and $B(\overline{r}) = \overline{B}$. Substituting into \eqnref{geodesic} and retaining terms to first order
\begin{equation}
\ddot{\delta} = - \frac{2\overline{A}^2L^2}{\overline{r}^3\overline{A}'\overline{B}}\left(\frac{\overline{A}''}{2\overline{A}^2} - \frac{{\overline{A}'}^2}{\overline{A}^3}\right)\delta + \frac{3L^2}{\overline{r}^4}\delta.
\end{equation}
Assuming a solution of form $\delta = \delta_0\cos(-i\Omega\tau)$,
\begin{equation}
\Omega^2 = \frac{L^2}{\overline{r}^3\overline{B}}\left(\frac{\overline{A}''}{\overline{A}'} - \frac{2\overline{A}'}{\overline{A}} + \frac{3}{\overline{r}}\right).
\end{equation}
We may rewrite the radial motion as
\begin{equation}
\widetilde{r} = \overline{r} + \delta_0\cos(-i\Omega\tau).
\end{equation}
If we compare this with an elliptic Keplerian orbit of small eccentricity $e$
\begin{align}
\widetilde{r} = {} & \frac{a(1 - e^2)}{1 + e\cos(\omega_0\tau)} \\
 = {} & a\left[1 - e\cos(\omega_0\tau) + \ldots \, \right]
\end{align}
to first order in $e$, where $a$ is the semimajor axis and $\omega_0$ is the orbital frequency; we may identify our perturbed orbit with an elliptical orbit where\cite{Kerner2001a}
\begin{equation}
\overline{r} = a; \quad \delta_0 = -ea.
\end{equation}
The eccentricity is the small parameter $|e| = |\delta_0/r| \ll 1$. To this accuracy one cannot distinguish between $a$ and the semilatus rectum $p = a(1 - e^2)$.

Unless $\omega_0 = \Omega$ the elliptical motion will be asynchronous with the orbital motion: there will be precession of the periapsis. The orbital frequency is
\begin{equation}
\omega_0^2 = \frac{L^2}{\overline{r}^4}.
\end{equation}
In one revolution the ellipse will precess about the focus by
\begin{align}
\varpi = {} & \omega_0\left(\frac{2\pi}{\Omega} - \frac{2\pi}{\omega_0}\right) \nonumber \\
 = {} & 2\pi\left(\frac{\omega_0}{\Omega} - 1\right)
\end{align}
The precession is cumulative, so a small deviation may be measurable over sufficient time.

For the $f(R)$ metric defined by equations \eqref{eq:A_metric} and \eqref{eq:B_metric} the epicyclic frequency is
\begin{equation}
\Omega^2 = \omega_0^2 \left[1 - \frac{3r\sub{S}}{\overline{r}} - \zeta(\Upsilon,r\sub{S},\overline{r})\right],
\end{equation}
defining the function
\begin{align}
\zeta(\Upsilon,r\sub{S},\overline{r}) = {} & r\sub{S}\left(\recip{3\overline{r}} + \Upsilon\right)\exp(-\Upsilon r) + \frac{\Upsilon^2\overline{r}^2\exp(-\Upsilon r)}{3 + (1 + \Upsilon \overline{r})\exp(-\Upsilon r)} \nonumber \\
 & {} \times \left\{1 - \frac{r\sub{S}}{\overline{r}}\left[1 + \frac{\exp(-\Upsilon r)}{3}\right] - \frac{\Upsilon r\sub{S}\exp(-\Upsilon r)}{3}\right\}.
\end{align}
This characterizes the deviation from the Schwarzschild case: the change in the precession per orbit relative to Schwarzschild is
\begin{align}
\upDelta \varpi = {} & \varpi - \varpi\sub{S} \\
 = {} & \pi\zeta,
\end{align}
using the subscript $\mathrm{S}$ to denote the Schwarzschild value. To obtain the last line we have expanded to lowest order, assuming that $\zeta$ is small.\footnote{There is one term in $\zeta$ that is not explicitly $\order{\varepsilon}$. Numerical evaluation shows that this is $< 0.6$ for the applicable range of parameters.} Since $\zeta \geq 0$, the precession rate is enhanced relative to GR.

Let us now apply this to the classic test of planetary precession in the solar system. \Tabref{Precess} shows the orbital properties of the planets. We will use the deviation in perihelion precession rate from the GR prediction to constrain the value of $\zeta$, and hence $\Upsilon$ and $a_2$.
\begin{table}[tbh]\footnotesize
\centering
\begin{tabular}{l D{.}{.}{2.8} D{.}{.}{3.8} r @{$\:\pm\:$} l D{.}{.}{1.8}}
\toprule
 & \multicolumn{1}{c}{Semimajor axis\cite{Cox2000}} & \multicolumn{1}{c}{Orbital period\cite{Cox2000}} & \multicolumn{2}{c}{Precession rate\cite{Pitjeva2009a}} & \multicolumn{1}{c}{Eccentricity\cite{Cox2000}} \\
Planet & \multicolumn{1}{c}{$r/\SI{e11}{\metre}$} & \multicolumn{1}{c}{$(2\pi/\omega_0)/\si{\yr}$} & \multicolumn{2}{c}{$\upDelta \varpi \pm \sigma_{\upDelta \varpi}/\si{\milli\as\per\yr}$} & \multicolumn{1}{c}{$e$} \\
\midrule
Mercury & 0.57909175 & 0.24084445 & $-0.040$ & $\phantom{0}0.050$ & 0.20563069 \\
Venus & 1.0820893 & 0.61518257 & $0.24\phantom{0}$ & $\phantom{0}0.33$ & 0.00677323 \\
Earth & 1.4959789 & 0.99997862 & $0.06\phantom{0}$ & $\phantom{0}0.07$ & 0.01671022 \\
Mars & 2.2793664 & 1.88071105 & $-0.07\phantom{0}$ & $\phantom{0}0.07$ & 0.09341233 \\
Jupiter & 7.7841202 & 11.85652502 & $0.67\phantom{0}$ & $\phantom{0}0.93$ & 0.04839266 \\
Saturn & 14.267254 & 29.42351935 & $-0.10\phantom{0}$ & $\phantom{0}0.15$ & 0.05415060 \\
Uranus & 28.709722 & 83.74740682 & $-38.9\phantom{00}$ & $39.0$ & 0.04716771 \\
Neptune & 44.982529 & 163.723204 & $-44.4\phantom{00}$ & $54.0$ & 0.00858587 \\
Pluto & 59.063762 & 248.0208 & $28.4\phantom{00}$ & $25.1$ & 0.24880766 \\
\bottomrule
\end{tabular}
\caption{Orbital properties of the eight major planets and Pluto. We take the semimajor orbital axis to be the flat-space distance $r$, not the coordinate $\widetilde{r}$. The eccentricity is not used in calculations, but is given to assess the accuracy of neglecting terms $\order{e^2}$.\label{tab:Precess}}
\end{table}
Since several of the deviations are negative, they cannot be explained by $f(R)$ corrections. This may be considered as evidence against $f(R)$-gravity; however, all the precession rates are consistent with GR predictions ($\upDelta \varpi = 0$), thus we cannot conclusively rule out $f(R)$-gravity. Since the deviations are zero to within their uncertainties, we may use the size of these uncertainties to constrain the $f(R)$ correction. \Tabref{Constraint} shows the constraints for $\Upsilon$ and $a_2$ obtained by equating the uncertainty in the precession rate $\sigma_{\upDelta \varpi}$ with the $f(R)$ correction, and similarly using twice the uncertainty $2\sigma_{\upDelta \varpi}$.
\begin{table}[bht]\footnotesize
\centering
\begin{tabular}{l D{.}{.}{2.2} D{.}{.}{5.1} D{.}{.}{2.2} D{.}{.}{5.1} }
\toprule
 &  \multicolumn{2}{c}{Using $\sigma_{\upDelta \varpi}$} & \multicolumn{2}{c}{Using $2\sigma_{\upDelta \varpi}$} \\
Planet & \multicolumn{1}{c}{$\Upsilon/\SI{e-11}{\per\metre}$} & \multicolumn{1}{c}{$|a_2|/\SI{e18}{\metre^2}$} & \multicolumn{1}{c}{$\Upsilon/\SI{e-11}{\per\metre}$} & \multicolumn{1}{c}{$|a_2|/\SI{e18}{\metre^2}$} \\
\midrule
Mercury & 52.6 & 1.2 & 51.3 & 1.3 \\
Venus & 25.3 & 5.2 & 24.6 & 5.5 \\
Earth & 19.1 & 9.1 & 18.6 & 9.6 \\
Mars & 12.2 & 22 & 11.9 & 24 \\
Jupiter & 2.96 & 380 & 2.87 & 410 \\
Saturn & 1.69 & 1200 & 1.63 & 1200 \\
Uranus & 0.58 & 9800 &  0.56 & 11000 \\
Neptune & 0.35 & 28000 & 0.33 & 31000 \\
Pluto & 0.26 & 49000 & 0.25 & 55000 \\
\bottomrule
\end{tabular}
\caption{Bounds calculated using uncertainties in planetary perihelion precession rates. $\Upsilon$ must be greater than or equal to the tabulated value, $|a_2|$ must be less than or equal to the tabulated value.\label{tab:Constraint}}
\end{table}

While the presence of negative deviations is evidence against $f(R)$-gravity, the tightest numerical constraint, obtained from the orbit of Mercury, is many orders of magnitude worse than obtained from laboratory tests in \secref{Fifth}. This bound is not much more stringent than the requirement that $\Upsilon r > 1$ over solar system scales. This is not surprising: for there to be a measurable precession effect the $f(R)$ modification to gravity must be significant; this implies that $\exp(-\Upsilon r)$ cannot be negligibly small.

\section{Discussion \& Remaining Questions}\label{sec:f_Discuss}

We have seen that gravitational radiation is modified in $f(R)$-gravity, as the Ricci scalar is no longer constrained to be zero. In linearized theory we find that there is an additional mode of oscillation, that of the Ricci scalar. However, based upon constraints from fifth-force experiments this mode seems unlikely to be excited in astrophysical processes. In $f(R)$ theory, the two transverse GW modes are modified from their GR counterparts to include a contribution from the Ricci scalar, see \eqnref{h_metric}, allowing us to probe the curvature of the strong-field regions from which GWs originate. However, further study is needed in order to understand how GW waves behave in a region with background curvature, in particular when $R$ is non-zero. This will be done in subsequent work.

Gravitational radiation is not the only way to test $f(R)$ theory. From linearized theory we have deduced the weak-field metrics for some simple mass distributions. These indicate that BH solutions are not the same as in GR. Using these weak-field results it is possible to constrain some parameters of $f(R)$. The strongest constraints come from fifth-force tests, but we have also derived the epicyclic frequency for near circular orbits. This is as an independent measurement, perhaps to check $f(R)$ in a different regime. We find that the current errors in planetary precession rates are too large to be explained by $f(R)$ modifications; they require $a_2$ to be unreasonably large. Additionally, some of the estimated deviations from GR precession rates are negative, which cannot be achieved with $f(R)$ corrections. Since all of the deviations are consistent with zero, we cannot use these as proof against $f(R)$, just that it does not modify gravity on solar system scales.

It is possible that $f(R)$-gravity is not universal --- that it is different in different regions of space. This could occur if $f(R)$ is just an approximate effective theory, then the range of a particular parametrization's applicability could be limited to a specific domain. For example, we could imagine that the effective theory in the vicinity of a massive BH where the curvature is large is different from in the solar system where curvature is small; alternatively $f(R)$ could evolve with cosmological epoch so that it varies with redshift.

Another possibility is that $f(R)$-gravity is modified in the presence of matter via the chameleon mechanism\cite{Khoury2004, Khoury2004a}. In metric $f(R)$ this corresponds to a nonlinear effect arising from a large departure of the Ricci scalar from its background value\cite{DeFelice2010}. The mass of the effective scalar degree of freedom then depends upon the density of its environment. In a region of high matter density, such as the Earth, the deviations from standard gravity would be exponentially suppressed due to a large effective $\Upsilon$; while on cosmological scales, where the density is low, the scalar would have a small $\Upsilon$, perhaps of the order $H_0/c$\cite{Khoury2004, Khoury2004a}. The chameleon mechanism allows $f(R)$ gravity to pass solar system tests while remaining of interest for cosmology. In the context of gravitational radiation, this would mean that the Ricci scalar mode could freely propagate on cosmological scales\cite{Corda2009}. Unfortunately, since the chameleon mechanism suppresses the effects of $f(R)$ in the presence of matter, this mode would have to be excited by something other than the movement of matter.

An obvious extension to the work presented here is to consider the case when $a_0$ is non-zero. We could then consider an expansion about (anti-)de Sitter space. This is interesting because the current $\Lambda$CDM paradigm indicates that we live in a universe with a positive cosmological constant\cite{Hinshaw2009}. 
